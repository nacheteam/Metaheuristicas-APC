\documentclass[12pt,a4paper]{article}
\usepackage[utf8]{inputenc}
\usepackage[spanish]{babel}
\usepackage{amsmath}
\usepackage{amsfonts}
\usepackage{amssymb}
\usepackage{graphicx}
\usepackage[left=2cm,right=2cm,top=2cm,bottom=2cm]{geometry}

%--------------------------------------------------------%
%         Paquetes usados sólo para esta entrega         %
%--------------------------------------------------------%

\usepackage[hidelinks]{hyperref}
\usepackage{algorithm}
\usepackage{algorithmic}
\usepackage{float}




\author{Ignacio Aguilera Martos \\
	DNI: 77448262V       e-mail: nacheteam@correo.ugr.es \\
	Grupo de prácticas 1 Lunes 17:30-19:30}
\title{Práctica Algoritmos de Trayectorias \\ Metaheurísticas}
\date{Curso 2017-2018}

%Quita la sangría
\setlength{\parindent}{0cm}


\begin{document}
	\maketitle

	\tableofcontents

	\newpage

	%p 56

%	\framebox[16cm][c]{\LaTeX}

	\section{Introducción del problema}
	\label{sec:introProblema}




	\section{Introducción de la práctica}
	\label{sec:introPractica}



	\newpage

	\section{Descripción común a todos los algoritmos}


	\newpage
	
	\section{Enfriamiento Simulado}
	\label{sec:ES}
	
	El algoritmo de enfriamiento simulado basa su comportamiento en varios factores numéricos entre los que podemos encontrar a la temperatura inicial, temperatura final y el esquema de enfriamiento que nos va a indicar cuanta exploración y explotación va a tener el algoritmo en función de la solución inicial conseguida.
	
	En primer lugar la temperatura inicial la calculamos con las constantes $\mu$ y $\phi$ que vienen determinadas mediante el guión por el valor 0.3 ambas. Esto nos indica que tenemos probabilidad 0.3 de aceptar una solución un 30\% peor que la que estamos considerando actualmente.
	
	A raíz de esto podemos definir la temperatura inicial como $T0 = \frac{\mu  \cdot C(S_0)}{-\log(\phi)}$ donde $C(S_0)$ es el coste de la solución inicial.
	
	Así mismo definimos la temperatura final como $TF=10^{-3}$. Como podemos tener la casuística de que desde el inicio del algoritmo la temperatura inicial ya sea menor que la final he definido la temperatura final como $TF=10^{-3}$ si esta constante es menor que la temperatura inicial o como $TF=T0-10^{-3}$ para tener así un margen de aplicación del algoritmo.
	
	El esquema de enfriamiento propuesto ha sido el esquema de enfriamiento de Cauchy modificado que nos otorga una convergencia mayor al decrementar la temperatura de forma más rápida que una lineal. Estas dos formas serán comparadas posteriormente en el análisis de resultados. El esquema de enfriamiento de Cauchy viene dado por las fórmulas:
	
	\vspace{10px}
	
	$\beta = \frac{T_0 - T_f}{M\cdot T_0\cdot T_f}$
	
	\vspace{10px}
	
	$T_{k+1} = \frac{T_k}{1+\beta \cdot T_k}$
	
	\vspace{10px}
	
	Donde M es el número de enfriamientos a realizar.
	
	A parte de estas constantes debemos tener en cuenta que el algoritmo está limitado a 15000 evaluaciones y el procedimiento de enfriamiento se aplica cuando hemos visitado $10\cdot D$ vecinos donde D es la dimensión del problema o cuando se han aceptado $0.1\cdot 10\cdot D$ vecinos como soluciones por el procedimiento.
	
	A continuación se describe el algoritmo en pseudocódigo: 
	
	\begin{algorithm}
		\caption{EnfriamientoSimulado(data,k,MAX\_EVALS)}
		\begin{algorithmic}
			\STATE ncar $\leftarrow$ número de características
			\STATE sol  $\leftarrow$ solución inicial aleatoria
			\STATE valoracion $\leftarrow$ valoración de la solución inicial
			\STATE mejor\_sol $\leftarrow$ sol
			\STATE valoracion\_mejor\_sol $\leftarrow$ valoracion
			\STATE
			\STATE $T_0$ $\leftarrow$ $\frac{\mu  \cdot C(S_0)}{-\log(\phi)}$
			\STATE
			\IF{T0$<$$10^{-3}$}
				\STATE $T_f$ $\leftarrow$ $10^{-3}$
			\ELSE
				\STATE $T_f$  $\leftarrow$ T0-$10^{-3}$
			\ENDIF
			\STATE max\_vecinos $\leftarrow$ 10*ncar
			\STATE M $\leftarrow$ $\frac{MAX\_EVALS}{max\_vecinos}$
			\STATE max\_exitos $\leftarrow$ 0.1*max\_vecinos
			\STATE
			\STATE $\beta$ $\leftarrow$ $\frac{T_0-T_f}{M*T_0*T_f}$
			\STATE
			\STATE t $\leftarrow$ $T_0$
			\STATE evaluaciones $\leftarrow$ 1
			\STATE K $\leftarrow$ 1
			\WHILE{t$>$$T_f$ and evaluaciones$<$MAX\_EVALS}
				\STATE vecinos $\leftarrow$ 0
				\STATE aceptados $\leftarrow$ 0
				\WHILE{aceptados$<$max\_exitos and vecinos$<$max\_vecinos}
					\STATE vecinos $\leftarrow$ vecinos + 1
					\STATE evaluaciones $\leftarrow$ evaluaciones + 1
					\STATE vecino $\leftarrow$ Mutación de una posición aleatoria
					\STATE valoracion\_vecino $\leftarrow$ Valoración del vecino
					\STATE delta $\leftarrow$ valoracion - valoracion\_vecino
					\IF{delta$<$0 or valorAleatorio(0,1)$<$exp(-delta/(t*K))}
						\STATE sol $\leftarrow$ vecino
						\STATE valoracion $\leftarrow$ valoracion\_vecino
						\STATE aceptados $\leftarrow$ aceptados + 1
						
						\IF{valoracion\_mejor\_sol$<$valoracion}
							\STATE mejor\_sol $\leftarrow$ sol
							\STATE valoracion\_mejor\_sol $\leftarrow$ valoracion
						\ENDIF
					\ENDIF
				\ENDWHILE
				\STATE t $\leftarrow$ t/(1.0+$\beta$*t)
				\STATE K $\leftarrow$ K+1
			\ENDWHILE
		\end{algorithmic}
	\end{algorithm}
	
	\newpage
	
	\section{Búsqueda Local Reiterada}
	\label{sec:ILS}
	
	La búsqueda local reiterada es un algoritmo que se basa en realizar mutaciones a una solución e ir aplicando el algoritmo de búsqueda local a estas soluciones mutadas quedándote con la mejor de ellas en el proceso.
	
	La intención de este algoritmo es realizar un reinicio controlado de las soluciones e ir aplicando la búsqueda local a estas soluciones pseudoaleatorias (no son aleatorias puras ya que son una mutación de una solución aleatoria mejorada con búsqueda local).
	
	Estas mutaciones permiten al algoritmo simple de búsqueda local escaparse de los máximos locales ya que en el proceso de mutación podemos obtener una solución peor que la actual y luego mejorarla mediante el uso de la búsqueda local.
	
	El procedimiento de mutación es similar al usado en los algoritmos genéticos y el usado durante todo el desarrollo de las tres prácticas.
	
	A continuación se describe en pseudocódigo las funciones empleadas en la mutación:
	
	\begin{algorithm}
		\caption{mutacionILS(solucion,MU=0,SIGMA=0.4)}
		\begin{algorithmic}
			\STATE num\_mutaciones $\leftarrow$ 0.1*longitud(solucion)
			\STATE sample $\leftarrow$ num\_mutaciones índices aleatorios sin repetición entre 0 y logitud(solucion)
			\FOR{i en sample}
				\STATE solucion $\leftarrow$ mutacion(solucion,i,MU,SIGMA)
			\ENDFOR		
			\RETURN solucion
		\end{algorithmic}
	\end{algorithm}
	
	\begin{algorithm}
		\caption{mutacion(w,pos,MU,SIGMA)}
		\begin{algorithmic}
			\STATE incremento $\leftarrow$ gauss(MU,SIGMA)
			\STATE w[pos] $\leftarrow$ w[pos] + incremento
			\IF{w[pos]$<$0}
				\STATE w[pos] $\leftarrow$ 0
			\ELSIF{w[pos]$>$1}
				\STATE w[pos] $\leftarrow$ 1
			\ENDIF
			\RETURN w
		\end{algorithmic}
	\end{algorithm}
	
	A continuación describo el pseudocódigo de ILS:
	
	\begin{algorithm}
		\caption{ILS(data,k,MAX\_EVALS)}
		\begin{algorithmic}			
			\STATE ncar $\leftarrow$ longitud(data[0])
			\STATE mejor\_solucion $\leftarrow$ [0,0,...,0]
			\STATE valoracion\_mejor\_solucion $\leftarrow$ 0
			\STATE evaluaciones $\leftarrow$ 1
			\WHILE{evaluaciones $<$ MAX\_EVALS}
				\STATE solucion $\leftarrow$ Solución aleatoria
				\STATE valoracion $\leftarrow$ Valoracion(solucion)
				\IF{valoracion$>$valoracion\_mejor\_solucion}
					\STATE mejor\_solucion $\leftarrow$ solucion
					\STATE valoracion\_mejor\_solucion $\leftarrow$ valoracion
				\ENDIF
				\STATE mejorada,ev $\leftarrow$ busquedaLocal(data,k,1000,solucion)
				\STATE evaluaciones $\leftarrow$ evaluciones + ev
				\STATE valoracion\_mejorada $\leftarrow$ Valoracion(mejorada)
				\IF{valoracion\_mejorada$>$valoracion\_mejor\_solucion}
					\STATE mejor\_solucion $\leftarrow$ mejorada
					\STATE valoracion\_mejor\_solucion $\leftarrow$ valoracion\_mejorada
				\ENDIF
				\IF{valoracion $>$ valoracion\_mejorada}
					\STATE mejor\_local $\leftarrow$ solucion
					\STATE valoracion\_mejor\_local $\leftarrow$ valoracion
				\ELSE
					\STATE mejor\_local $\leftarrow$ mejorada
					\STATE valoracion\_mejor\_local $\leftarrow$ valoracion\_mejorada
				\ENDIF
				\STATE evaluaciones $\leftarrow$ evaluaciones + 1
				\IF{valoracion\_mejor\_local$>$valoracion\_mejor\_solucion}
					\STATE mejor\_solucion $\leftarrow$ mejor\_local
					\STATE valoracion\_mejor\_solucion $\leftarrow$ valoracion\_mejor\_local
				\ENDIF
				\FOR{i en [0,...,13]}
					\STATE mutada $\leftarrow$ mutacionILS(mejor\_local)
					\STATE valoracion\_mutada $\leftarrow$ Valoracion(mutada)
					\STATE mutada\_mejorada,ev $\leftarrow$ busquedaLocal(data,k,1000,mutada)
					\STATE valoracion\_mutada\_mejorada $\leftarrow$ Valoracion(mutada\_mejorada)
					\STATE evaluaciones $\leftarrow$ evaluaciones + 2 + ev
					\IF{valoracion\_mutada$>$valoracion\_mutada\_mejorada}
						\STATE mejor\_local $\leftarrow$ mutada
						\STATE valoracion\_mejor\_local $\leftarrow$ valoracion\_mutada
					\ELSE
						\STATE mejor\_local $\leftarrow$ mutada\_mejorada
						\STATE valoracion\_mejor\_local $\leftarrow$ valoracion\_mutada\_mejorada
					\ENDIF
					
					\IF{valoracion\_mejor\_local$>$valoracion\_mejor\_solucion}
						\STATE mejor\_solucion $\leftarrow$ mejor\_local
						\STATE valoracion\_mejor\_solucion $\leftarrow$ valoracion\_mejor\_local
					\ENDIF
				\ENDFOR
			\ENDWHILE
			\RETURN mejor\_solucion
			
		\end{algorithmic}
	\end{algorithm}
	
	\section{Evolución Diferencial}
	\label{sec:DE}

	\section{Pseudocódigo Genético Estacionario}
	\label{sec:GE}


	\begin{algorithm}
		\caption{GeneticoEstacionario(data,k,operador\_cruce)}
		\begin{algorithmic}
			\STATE num\_padres $\leftarrow$ 0
			\IF{operador\_cruce == cruceAritmetico}
				\STATE num\_padres $\leftarrow$ 4
			\ELSIF{operador\_cruce == cruceBLX}
				\STATE num\_padres $\leftarrow$ 2
			\ELSE
				\STATE Error en el operador de cruce.
			\ENDIF
			\STATE
			\STATE poblacion $\leftarrow$ generaPoblacionInicial(numero\_caracteristicas)
			\STATE valoraciones $\leftarrow$ tasa\_agregada + tasa\_reduccion de cada individuo de la poblacion
			\STATE evaluaciones $\leftarrow$ TAM\_POBLACION
			\WHILE{evaluaciones $<$ MAX\_EVALUACIONES}
				\STATE padres $\leftarrow$ Padres escogidos por torneo binario según num\_padres
				\STATE hijos $\leftarrow$ Obtenemos los hijos según operador\_cruce con los padres calculados.
				\STATE
				\STATE Muta cada gen de los hijos si uniforme(0,1) es menor que 0.001.
				\STATE poblacion $\leftarrow$ [poblacion,hijos]
				\STATE valoraciones $\leftarrow$ [valoraciones,valoraciones de los hijos]
				\STATE Obtener los índices que los 30 mejores individuos de la población y quedarse con ellos.
				\STATE Actualizar poblacion y valoraciones según los índices obtenidos.
				\STATE evaluaciones $\leftarrow$ evaluaciones+2
			\ENDWHILE
			\RETURN Devolver al individuo con mayor valoración de la población.
		\end{algorithmic}
	\end{algorithm}

	\newpage

	\section{Peseudocódigo Genético Generacional}
	\label{sec:GG}

	\begin{algorithm}
		\caption{GeneticoGeneracional(data,k,operador\_cruce)}
		\begin{algorithmic}
			\STATE poblacion $\leftarrow$ generaPoblacionInicial(num\_caracteristicas)
			\STATE mutaciones $\leftarrow$ PROB\_MUTACION*TAM\_POBLACION*num\_caracteristicas
			\STATE num\_parejas $\leftarrow$ TAM\_POBLACION*PROB\_CRUCE
			\STATE valoraciones $\leftarrow$ valoraciones de la población
			\STATE mejor\_solucion $\leftarrow$ Mejor solución de la población.
			\WHILE{evaluaciones $<$ MAX\_EVALUACIONES}
				\STATE hijos $\leftarrow$ [ ]
				\FOR{i=0 , ... , num\_parejas-1}
					\IF{operador\_cruce==cruceAritmetico}
						\STATE padres $\leftarrow$ genera 4 padres con torneoBinario
						\STATE hijos $\leftarrow$ [hijos,operador\_cruce(padres[0],padres[1])]
						\STATE hijos $\leftarrow$ [hijos,operador\_cruce(padres[2],padres[3])]
						\STATE hijos $\leftarrow$ [hijos,operador\_cruce(padres[0],padres[2])]
						\STATE hijos $\leftarrow$ [hijos,operador\_cruce(padres[1],padres[2])]
					\ELSE
						\STATE padres $\leftarrow$ genera 2 padres con torneoBinario
						\STATE hijos $\leftarrow$ [hijos,operador\_cruce(padres[0],padres[1])]
						\STATE hijos $\leftarrow$ [hijos,operador\_cruce(padres[0],padres[1])]
					\ENDIF
				\ENDFOR
				\STATE Muta la nueva población de hijos con probabilidad 0.001 con una distribución $gauss(\mu=0,\sigma=0.3)$
				\STATE Rellena la población de hijos con padres haciendo torneos binarios.
				\STATE poblacion $\leftarrow$ hijos
				\STATE Actualiza las valoraciones de los individuos.
				\STATE Si el peor de la nueva población es peor que el mejor de la anterior lo sustituimos.
				\STATE Actualiza el mejor de la población.
			\ENDWHILE
			\RETURN Mejor de la población.
		\end{algorithmic}
	\end{algorithm}

	\newpage

	\section{Pseudocódigo Meméticos}
	\label{sec:memeticos}

	\begin{algorithm}
		\caption{Memetico(data,k,operador\_cruce,nGeneraciones,prob\_bl,mejores=False)}
		\begin{algorithmic}
			\STATE poblacion $\leftarrow$ generaPoblacionInicial(num\_caracteristicas)
			\STATE mutaciones $\leftarrow$ PROB\_MUTACION*TAM\_POBLACION*num\_caracteristicas
			\STATE num\_parejas $\leftarrow$ TAM\_POBLACION*PROB\_CRUCE
			\STATE valoraciones $\leftarrow$ valoraciones de la población
			\STATE mejor\_solucion $\leftarrow$ Mejor solución de la población.
			\STATE contador\_generaciones $\leftarrow$ 1
			\WHILE{evaluaciones $<$ MAX\_EVALUACIONES}
				\IF{contador\_generaciones\%nGeneraciones==0}
					\STATE n\_elem\_bl $\leftarrow$ prob\_bl*TAM\_POBLACION
					\STATE individuos $\leftarrow$ [ ]
					\IF{not mejores}
						\STATE individuos $\leftarrow$ Tomar n\_elem\_bl de forma aleatoria desde 0,...,TAM\_POBLACION-1
					\ELSE
						\STATE individuos $\leftarrow$ Toma los 0.1*TAM\_POBLACION mejores de la poblacion
					\ENDIF
					\FOR{ind en individuos}
						\STATE Aplica la búsqueda local a poblacion[ind]
						\STATE Actualiza el número de evaluaciones.
					\ENDFOR
					\STATE Actualiza las valoraciones
					\STATE Actualiza las evaluaciones.
				\ENDIF
				\STATE
				\STATE hijos $\leftarrow$ [ ]
				\FOR{i=0 , ... , num\_parejas-1}
					\IF{operador\_cruce==cruceAritmetico}
						\STATE padres $\leftarrow$ genera 4 padres con torneoBinario
						\STATE hijos $\leftarrow$ [hijos,operador\_cruce(padres[0],padres[1])]
						\STATE hijos $\leftarrow$ [hijos,operador\_cruce(padres[2],padres[3])]
						\STATE hijos $\leftarrow$ [hijos,operador\_cruce(padres[0],padres[2])]
						\STATE hijos $\leftarrow$ [hijos,operador\_cruce(padres[1],padres[2])]
					\ELSE
						\STATE padres $\leftarrow$ genera 2 padres con torneoBinario
						\STATE hijos $\leftarrow$ [hijos,operador\_cruce(padres[0],padres[1])]
						\STATE hijos $\leftarrow$ [hijos,operador\_cruce(padres[0],padres[1])]
					\ENDIF
				\ENDFOR
				\STATE Muta la nueva población de hijos con probabilidad 0.001 con una distribución $gauss(\mu=0,\sigma=0.3)$
				\STATE Rellena la población de hijos con padres haciendo torneos binarios.
				\STATE poblacion $\leftarrow$ hijos
				\STATE Actualiza las valoraciones de los individuos.
				\STATE Si el peor de la nueva población es peor que el mejor de la anterior lo sustituimos.
				\STATE Actualiza el mejor de la población.
				\STATE contador\_generaciones $\leftarrow$ contador\_generaciones + 1
			\ENDWHILE
			\RETURN Mejor de la población.
		\end{algorithmic}
	\end{algorithm}

	Donde prob\_bl es el porcentaje de la población al que queremos aplicar la búsqueda local.

	\newpage

	\section{Pseudocódigo KNN}
	\label{sec:knn}

	\begin{algorithm}
		\caption{KNN(w,datos\_test,datos\_entrenamiento, etiquetas\_entrenamiento, etiquetas\_test, k, mismos\_conjuntos)}
		\begin{algorithmic}
			\STATE tam\_datos\_entrenamiento $\leftarrow$ longitud(datos\_entrenamiento)
			\STATE clases $\leftarrow$ []
			\FOR{i=0,...,longitud(datos\_test)}
			\STATE p $\leftarrow$ datos\_test[i]
			\STATE w\_m $\leftarrow$ Repetir el vector w tantas veces como datos haya en datos\_entrenamiento.
			\STATE p\_m $\leftarrow$ Repetir el vector p tantas veces como datos haya en datos\_entrenamiento.
			\STATE dist $\leftarrow$ $w\_m \cdot (p\_m - datos\_entrenamiento)^2$
			\IF{mismos\_conjuntos}
			\STATE dist[i] $\leftarrow$ $\infty$
			\ENDIF
			\STATE mins $\leftarrow$ Los k índices correspondientes a las distancias más pequeñas.
			\STATE clases $\leftarrow$ [clases, masComun(etiquetas\_entrenamiento[mins])]
			\ENDFOR
			\RETURN $\frac{Numero \ de \ elementos \ de \ clases \ que \ han \ acertado \ con \ respecto \ a \ etiquetas\_test}{longitud(etiquetas\_test)}$
		\end{algorithmic}
	\end{algorithm}

	Cabe notar que el número que devolvemos está entre 0 y 1, por lo que en los algoritmos de valoración debemos tener esto en cuenta para multiplicarlo por 100 y convertirlo en un porcentaje.

	\newpage

	\section{Pseudocódigo Relief}
	\label{sec:relief}

	\begin{algorithm}
		\caption{elementoMinimaDistancia(e,lista)}
		\begin{algorithmic}
			\STATE distancias $\leftarrow$ [ ]
			\FOR{l en lista}
			\IF{l!=e}
			\STATE distancias $\leftarrow$ [distancias, distancia(e,l,[1..1])]
			\ELSE
			\STATE distancias $\leftarrow$ [distancias, max(distancias)]
			\ENDIF
			\ENDFOR
			\STATE indice\_menor\_distancia $\leftarrow$ índice del elemento de menor valor del vector distancias.
			\RETURN lista[indice\_menor\_distancia]
		\end{algorithmic}
	\end{algorithm}

	\begin{algorithm}
		\caption{Relief(data)}
		\begin{algorithmic}
			\STATE w $\leftarrow$ vector de pesos a 0
			\FOR{elemento en data}
			\STATE clase $\leftarrow$ clase de elemento
			\STATE amigos $\leftarrow$ [ ]
			\STATE enemigos $\leftarrow$ [ ]
			\FOR{e en data}
			\IF{e!=elemento AND e[longitud(e)-1]==clase}
			\STATE amigos $\leftarrow$ [amigos, e]
			\ELSE
			\STATE enemigos $\leftarrow$ [enemigos, e]
			\ENDIF
			\ENDFOR
			\STATE amigo\_cercano $\leftarrow$ elementoMinimaDistancia(elemento, amigos)
			\STATE enemigo\_cercano $\leftarrow$ elementoMinimaDistancia(elemento, enemigos)
			\STATE resta\_enemigo $\leftarrow$ element-enemigo\_cercano
			\STATE resta\_amigo $\leftarrow$ element-amigo\_cercano
			\STATE w $\leftarrow$ w + resta\_enemigo - resta\_amigo
			\STATE $w_{max}$ $\leftarrow$ máximo de w
			\ENDFOR
			\FOR{i en [0..longitud(w)-1]}
			\IF{w[i]$<$0}
			\STATE w[i] $\leftarrow$ 0
			\ELSE
			\STATE w[i] $\leftarrow$ $\frac{w[i]}{w_{max}}$
			\ENDIF
			\ENDFOR
			\RETURN w
		\end{algorithmic}
	\end{algorithm}

	\newpage

	\section{Pseudocódigo Búsqueda Local}
	\label{sec:bl}

	\begin{algorithm}
		\caption{primerVector(n)}
		\begin{algorithmic}
			\STATE w $\leftarrow$ [ ]
			\FOR{i en [0..n-1]}
			\STATE w $\leftarrow$ [w, random.uniforme(0,1)]
			\ENDFOR
			\RETURN w
		\end{algorithmic}
	\end{algorithm}

	\begin{algorithm}
		\caption{busquedaLocal(data,k)}
		\begin{algorithmic}
			\STATE MAX\_EVALUACIONES $\leftarrow$ 15000
			\STATE MAX\_VECINOS $\leftarrow$ $20\cdot longitud(data[0])$
			\STATE vecinos $\leftarrow$ 0
			\STATE evaluaciones $\leftarrow$ 0
			\STATE posicion\_mutacion $\leftarrow$ 0
			\STATE w $\leftarrow$ primerVector(longitud(data[0]))
			\STATE valoracion\_actual $\leftarrow$ Valoracion(data,data,k,w)
			\WHILE{evaluaciones$<$MAX\_EVALUACIONES AND vecinos$<$MAX\_VECINOS}
				\STATE evaluaciones $\leftarrow$ evaluaciones+1
				\STATE vecinos $\leftarrow$ vecinos+1
				\STATE vecino, posicion\_mutacion $\leftarrow$ mutacion(w,posicion\_mutacion)
				\STATE valoracion\_vecino $\leftarrow$ Valoracion(data,data,k,vecino)
				\IF{valoracion\_vecino$>$valoracion\_actual}
					\STATE vecinos $\leftarrow$ 0
					\STATE w $\leftarrow$ vecino
					\STATE valoracion\_actual $\leftarrow$ valoracion\_vecino
					\STATE posicion\_mutacion $\leftarrow$ 0
				\ELSIF{posicion\_mutacion==longitud(w)}
					\STATE posicion\_mutacion $\leftarrow$ 0
				\ENDIF
			\ENDWHILE
			\RETURN w
		\end{algorithmic}
	\end{algorithm}

	\newpage

	\section{Procedimiento de desarrollo de la práctica}
	\label{sec:procedimiento}



	\newpage

	\section{Resultados}
	\label{sec:resultados}

	\begin{table}[ht]
		\centering
		\resizebox{\textwidth}{!}{
			\begin{tabular}{| c | c | c | c | c | c | c | c | c | c | c | c | c |}
				\cline{2-13}
				\multicolumn{1}{c|}{} & \multicolumn{4}{|c|}{Ozone} & \multicolumn{4}{| c|}{Parkinsons} & \multicolumn{4}{|c|}{Spectf-Heart} \\ [0.5ex]
				\cline{2-13}
				\multicolumn{1}{c|}{} & \%\_clas & \%\_red & Agr. & T (seg) & \%\_clas & \%\_red & Agr. & T  (seg) & \%\_clas & \%\_red & Agr. & T  (seg) \\ [0.5ex] \hline
				Partición 1 & 71.8750 & 0.0000 & 35.9375 & 0.0111 & 76.3158 & 0.0000 & 38.1579 & 0.0032 & 70.5882 & 0.0000 & 35.2941 & 0.0072 \\ [0.5ex] \hline
				Partición 2 & 84.3750 & 0.0000 & 42.1875 & 0.0097 & 81.5789 & 0.0000 & 40.7895 & 0.0046 & 77.9412 & 0.0000 & 38.9706 & 0.0092 \\ [0.5ex] \hline
				Partición 3 & 71.8750 & 0.0000 & 35.9375 & 0.0095 & 94.7368 & 0.0000 & 47.3684 & 0.0031 & 67.6471 & 0.0000 & 33.8235 & 0.0083 \\ [0.5ex] \hline
				Partición 4 & 81.2500 & 0.0000 & 40.6250 & 0.0088 & 73.6842 & 0.0000 & 36.8421 & 0.0029 & 60.2941 & 0.0000 & 30.1471 & 0.0072 \\ [0.5ex] \hline
				Partición 5 & 85.9375 & 0.0000 & 42.9688 & 0.0089 & 76.7442 & 0.0000 & 38.3721 & 0.0032 & 66.2338 & 0.0000 & 33.1169 & 0.0071 \\ [0.5ex] \hline
				Media & 79.0625 & 0.0000 & 39.5313 & 0.0096 & 80.6120 & 0.0000 & 40.3060 & 0.0034 & 68.5409 & 0.0000 & 34.2704 & 0.0078 \\ [0.5ex] \hline
			\end{tabular}
		}
		\label{tabla1NN}
		\caption{Resultados 1NN}

	\end{table}



	\begin{table}[h!]
		\centering
		\resizebox{\textwidth}{!}{
			\begin{tabular}{| c | c | c | c | c | c | c | c | c | c | c | c | c |}
				\cline{2-13}
				\multicolumn{1}{c|}{} & \multicolumn{4}{|c|}{Ozone} & \multicolumn{4}{| c|}{Parkinsons} & \multicolumn{4}{|c|}{Spectf-Heart} \\ [0.5ex]
				\cline{2-13}
				\multicolumn{1}{c|}{} & \%\_clas & \%\_red & Agr. & T (seg) & \%\_clas & \%\_red & Agr. & T  (seg) & \%\_clas & \%\_red & Agr. & T  (seg) \\ [0.5ex] \hline
				Partición 1 & 64.0625 & 0.0000 & 32.0313 & 2.3071 & 76.3158 & 0.0000 & 38.1579 & 0.2821 & 26.4706 & 0.0000 & 13.2353 & 1.0612 \\ [0.5ex] \hline
				Partición 2 & 85.9375 & 0.0000 & 42.9688 & 1.6189 & 60.5263 & 0.0000 & 30.2632 & 0.2874 & 73.5294 & 0.0000 & 36.7647 & 1.0387 \\ [0.5ex] \hline
				Partición 3 & 75.0000 & 0.0000 & 37.5000 & 1.7513 & 76.3158 & 0.0000 & 38.1579 & 0.2792 & 73.5294 & 0.0000 & 36.7647 & 1.1519 \\ [0.5ex] \hline
				Partición 4 & 73.4375 & 0.0000 & 36.7188 & 1.7399 & 78.9474 & 0.0000 & 39.4737 & 0.2807 & 73.5294 & 0.0000 & 36.7647 & 1.0449 \\ [0.5ex] \hline
				Partición 5 & 81.2500 & 0.0000 & 40.6250 & 1.6257 & 72.0930 & 0.0000 & 36.0465 & 0.2769 & 29.8701 & 0.0000 & 14.9351 & 0.6444 \\ [0.5ex] \hline
				Media & 75.9375 & 0.0000 & 37.9688 & 1.8086 & 72.8397 & 0.0000 & 36.4198 & 0.2812 & 55.3858 & 0.0000 & 27.6929 & 0.9882 \\ [0.5ex] \hline
			\end{tabular}
		}
		\label{tablaReliefK1}
		\caption{Resultados Relief con K=1}
	\end{table}



	\begin{table}[h!]
		\centering
		\resizebox{\textwidth}{!}{
			\begin{tabular}{| c | c | c | c | c | c | c | c | c | c | c | c | c |}
				\cline{2-13}
				\multicolumn{1}{c|}{} & \multicolumn{4}{|c|}{Ozone} & \multicolumn{4}{| c|}{Parkinsons} & \multicolumn{4}{|c|}{Spectf-Heart} \\ [0.5ex]
				\cline{2-13}
				\multicolumn{1}{c|}{} & \%\_clas & \%\_red & Agr. & T (seg) & \%\_clas & \%\_red & Agr. & T  (seg) & \%\_clas & \%\_red & Agr. & T  (seg) \\ [0.5ex] \hline
				Partición 1 & 76.5625 & 25.0000 & 50.7813 & 89.7886 & 78.9474 & 18.1818 & 48.5646 & 7.7242 & 72.0588 & 29.5454 & 50.8021 & 24.4655 \\ [0.5ex] \hline
				Partición 2 & 79.6875 & 15.2778 & 47.4826 & 75.9767 & 86.8421 & 36.3636 & 61.6029 & 7.4996 & 73.5294 & 20.4545 & 46.9920 & 44.5414 \\ [0.5ex] \hline
				Partición 3 & 68.7500 & 34.7222 & 51.7361 & 85.7498 & 94.7368 & 22.7273 & 58.7321 & 5.4946 & 75.0000 & 25.0000 & 50.0000 & 32.4179 \\ [0.5ex] \hline
				Partición 4 & 81.2500 & 27.7778 & 54.5139 & 93.7621 & 76.3158 & 13.6364 & 44.9761 & 6.5065 & 58.8235 & 29.5454 & 44.1845 & 59.5462 \\ [0.5ex] \hline
				Partición 5 & 78.1250 & 23.6111 & 50.8681 & 105.6183 & 76.7442 & 13.6364 & 45.1903 & 5.4971 & 62.3377 & 25.0000 & 43.6688 & 22.3619 \\ [0.5ex] \hline
				Media & 76.875 & 25.2778 & 51.0764 & 90.1781 & 82.7173 & 20.9091 & 51.8132 & 6.5444 & 68.3499 & 25.9091 & 47.1295 & 36.6666 \\ [0.5ex] \hline
			\end{tabular}
		}
		\label{tablaBLK1}
		\caption{Resultados Búsqueda Local con K=1}
	\end{table}


	\begin{table}[h!]
		\centering
		\resizebox{\textwidth}{!}{
			\begin{tabular}{| c | c | c | c | c | c | c | c | c | c | c | c | c |}
				\cline{2-13}
				\multicolumn{1}{c|}{} & \multicolumn{4}{|c|}{Ozone} & \multicolumn{4}{| c|}{Parkinsons} & \multicolumn{4}{|c|}{Spectf-Heart} \\ [0.5ex]
				\cline{2-13}
				\multicolumn{1}{c|}{} & \%\_clas & \%\_red & Agr. & T (seg) & \%\_clas & \%\_red & Agr. & T  (seg) & \%\_clas & \%\_red & Agr. & T  (seg) \\ [0.5ex] \hline
				Partición 1 & 76.5625 & 52.7778 & 64.6701 & 534.7600 & 73.6842 & 72.7273 & 73.2057 & 144.2443 & 73.5294 & 59.0909 & 66.3102 & 268.5947 \\ [0.5ex] \hline
				Partición 2 & 87.5000 & 51.3889 & 69.4444 & 482.7369 & 71.0526 & 68.1818 & 69.6172 & 145.8859 & 70.5882 & 54.5454 & 62.5668 & 343.7157 \\ [0.5ex] \hline
				Partición 3 & 76.5625 & 50.0000 & 63.2813 & 452.4577 & 97.3684 & 68.1818 & 82.7751 & 189.4074 & 72.0588 & 63.6364 & 67.8476 & 376.4886 \\ [0.5ex] \hline
				Partición 4 & 82.8125 & 55.5556 & 69.1840 & 481.0927 & 60.5263 & 68.1818 & 64.3541 & 192.2663 & 61.7647 & 61.3636 & 61.5642 & 353.3151 \\ [0.5ex] \hline
				Partición 5 & 79.6875 & 52.7778 & 66.2326 & 476.6389 & 74.4186 & 72.7273 & 73.5729 & 185.5641 & 75.3247 & 61.3636 & 68.3442 & 241.8399 \\ [0.5ex] \hline
				Media & 80.6250 & 52.5000 & 66.5625 & 485.5372 & 75.4100 & 70.0000  72.7050 & 171.4736 & 70.6532 & 60.0000 & 65.3266 & 316.7908 \\ [0.5ex] \hline
			\end{tabular}
		}
		\label{tablaAGEBLXK1}
		\caption{Resultados AGE-BLX con K=1}
	\end{table}

	\begin{table}[h!]
		\centering
		\resizebox{\textwidth}{!}{
			\begin{tabular}{| c | c | c | c | c | c | c | c | c | c | c | c | c |}
				\cline{2-13}
				\multicolumn{1}{c|}{} & \multicolumn{4}{|c|}{Ozone} & \multicolumn{4}{| c|}{Parkinsons} & \multicolumn{4}{|c|}{Spectf-Heart} \\ [0.5ex]
				\cline{2-13}
				\multicolumn{1}{c|}{} & \%\_clas & \%\_red & Agr. & T (seg) & \%\_clas & \%\_red & Agr. & T  (seg) & \%\_clas & \%\_red & Agr. & T  (seg) \\ [0.5ex] \hline
				Partición 1 & 78.1250 & 76.3889 & 77.2569 & 471.0574 & 68.4211 & 68.1818 & 68.3014 & 193.6963 & 76.4706 & 68.1818 & 72.3262 & 268.5880 \\ [0.5ex] \hline
				Partición 2 & 84.3750 & 76.3889 & 80.3819 & 476.7666 & 73.6842 & 63.6364 & 68.6603 & 190.7289 & 70.5882 & 86.3636 & 78.4759 & 342.4545 \\ [0.5ex] \hline
				Partición 3 & 81.2500 & 65.2778 & 73.2639 & 467.5143 & 78.9474 & 86.3636 & 82.6555 & 194.1895 & 70.5882 & 79.5455 & 75.0668 & 391.2035 \\ [0.5ex] \hline
				Partición 4 & 82.8125 & 66.6667 & 74.7396 & 476.7075 & 65.7895 & 77.2727 & 71.5311 & 194.0040 & 69.1176 & 70.4545 & 69.7861 & 357.9221 \\ [0.5ex] \hline
				Partición 5 & 76.5625 & 63.8889 & 70.2257 & 476.0006 & 76.7442 & 63.6364 & 70.1903 & 189.8313 & 74.0259 & 72.7273 & 73.3766 & 247.7655 \\ [0.5ex] \hline
				Media & 80.6250 & 69.7222 & 75.1736 & 473.6093 & 72.7173 & 71.8182 & 72.2677 & 192.4899 & 72.1581 & 75.4545 & 73.8063 & 321.5867 \\ [0.5ex] \hline
			\end{tabular}
		}
		\label{tablaAGECAK1}
		\caption{Resultados AGE-CA con K=1}
	\end{table}

	\begin{table}[h!]
		\centering
		\resizebox{\textwidth}{!}{
			\begin{tabular}{| c | c | c | c | c | c | c | c | c | c | c | c | c |}
				\cline{2-13}
				\multicolumn{1}{c|}{} & \multicolumn{4}{|c|}{Ozone} & \multicolumn{4}{| c|}{Parkinsons} & \multicolumn{4}{|c|}{Spectf-Heart} \\ [0.5ex]
				\cline{2-13}
				\multicolumn{1}{c|}{} & \%\_clas & \%\_red & Agr. & T (seg) & \%\_clas & \%\_red & Agr. & T  (seg) & \%\_clas & \%\_red & Agr. & T  (seg) \\ [0.5ex] \hline
				Partición 1 & 71.8750 & 45.8333 & 58.8542 & 649.8476 & 76.3158 & 45.4545 & 60.8852 & 269.7200 & 67.6471 & 47.7273 & 57.6872 & 387.5749 \\ [0.5ex] \hline
				Partición 2 & 78.1250 & 45.8333 & 61.9792 & 656.0903 & 86.8421 & 50.0000 & 68.4211 & 273.4718 & 70.5882 & 50.0000 & 60.2941 & 491.5933 \\ [0.5ex] \hline
				Partición 3 & 71.8750 & 44.4444 & 58.1597 & 645.3196 & 86.8421 & 50.0000 & 68.4211 & 271.0081 & 70.5882 & 50.0000 & 60.2941 & 533.1007 \\ [0.5ex] \hline
				Partición 4 & 81.2500 & 38.8889 & 60.0694 & 657.0321 & 78.9474 & 45.4545 & 62.2010 & 270.4252 & 67.6471 & 50.0000 & 58.8235 & 518.8346 \\ [0.5ex] \hline
				Partición 5 & 79.6875 & 38.8889 & 59.2882 & 657.8484 & 65.1163 & 54.5455 & 59.8309 & 261.5370 & 64.9351 & 45.4545 & 55.1948 & 376.8476 \\ [0.5ex] \hline
				Media & 76.5625 & 42.7778 & 59.6701 & 653.2276 & 78.8127 & 49.0909 & 63.9518 & 269.2324 & 68.2811 & 48.6364 & 58.4587 & 461.5902 \\ [0.5ex] \hline
			\end{tabular}
		}
		\label{tablaAGGBLXK1}
		\caption{Resultados AGG-BLX con K=1}
	\end{table}

	\begin{table}[h!]
		\centering
		\resizebox{\textwidth}{!}{
			\begin{tabular}{| c | c | c | c | c | c | c | c | c | c | c | c | c |}
				\cline{2-13}
				\multicolumn{1}{c|}{} & \multicolumn{4}{|c|}{Ozone} & \multicolumn{4}{| c|}{Parkinsons} & \multicolumn{4}{|c|}{Spectf-Heart} \\ [0.5ex]
				\cline{2-13}
				\multicolumn{1}{c|}{} & \%\_clas & \%\_red & Agr. & T (seg) & \%\_clas & \%\_red & Agr. & T  (seg) & \%\_clas & \%\_red & Agr. & T  (seg) \\ [0.5ex] \hline
				Partición 1 & 78.1250 & 44.4444 & 61.2847 & 1295.8234 & 86.8421 & 31.8182 & 59.3301 & 549.0934 & 69.1176 & 54.5455 & 61.8316 & 839.1870 \\ [0.5ex] \hline
				Partición 2 & 84.3750 & 50.0000 & 67.1875 & 1308.7905 & 84.2105 & 40.9091 & 62.5598 & 554.3215 & 70.5882 & 45.4545 & 58.0214 & 1027.5440 \\ [0.5ex] \hline
				Partición 3 & 79.6875 & 54.1667 & 66.9271 & 1287.9295 & 86.8421 & 31.8182 & 59.3301 & 459.0695 & 67.6471 & 56.8182 & 62.2326 & 1115.2930 \\ [0.5ex] \hline
				Partición 4 & 76.5625 & 41.6667 & 59.1146 & 1307.3627 & 81.5789 & 40.9091 & 61.2440 & 430.1995 & 72.0588 & 54.5454 & 63.3021 & 1053.8761 \\ [0.5ex] \hline
				Partición 5 & 81.2500 6 41.6667 & 61.4583 & 1307.9415 & 79.0698 & 36.3636 & 57.7167 & 405.0435 & 70.1299 & 45.4545 & 57.7922 & 662.3548 \\ [0.5ex] \hline
				Media & 80.0000 & 46.3889 & 63.1944 & 1301.5695 & 83.7087 & 36.3636 & 60.0362 & 479.5455 & 69.9083 & 51.3636 & 60.6360 & 939.6510 \\ [0.5ex] \hline
			\end{tabular}
		}
		\label{tablaAGGCAK1}
		\caption{Resultados AGG-CA con K=1}
	\end{table}

	\begin{table}[h!]
		\centering
		\resizebox{\textwidth}{!}{
			\begin{tabular}{| c | c | c | c | c | c | c | c | c | c | c | c | c |}
				\cline{2-13}
				\multicolumn{1}{c|}{} & \multicolumn{4}{|c|}{Ozone} & \multicolumn{4}{| c|}{Parkinsons} & \multicolumn{4}{|c|}{Spectf-Heart} \\ [0.5ex]
				\cline{2-13}
				\multicolumn{1}{c|}{} & \%\_clas & \%\_red & Agr. & T (seg) & \%\_clas & \%\_red & Agr. & T  (seg) & \%\_clas & \%\_red & Agr. & T  (seg) \\ [0.5ex] \hline
				Partición 1 & 76.5625 & 30.5556 & 53.5590 & 483.3842 & 84.2105 & 45.4545 & 64.8325 & 158.4467 & 75.0000 & 34.0909 & 54.5455 & 282.0805 \\ [0.5ex] \hline
				Partición 2 & 87.5000 & 34.7222 & 61.1111 & 491.0853 & 81.5789 & 40.9091 & 61.2440 & 155.5619 & 82.3529 & 34.0909 & 58.2219 & 358.5910 \\ [0.5ex] \hline
				Partición 3 & 75.0000 & 30.5556 & 52.7778 & 480.3444 & 89.4737 & 31.8182 & 60.6459 & 158.6902 & 82.3529 & 40.9091 & 61.6310 & 390.2710 \\ [0.5ex] \hline
				Partición 4 & 85.9375 & 34.7222 & 60.3299 & 492.3788 & 71.0526 & 31.8182 & 51.4354 & 165.8306 & 60.2941 & 34.0909 & 47.1925 & 362.2775 \\ [0.5ex] \hline
				Partición 5 & 85.9375 & 31.9444 & 58.9410 & 491.1975 & 72.0930 & 36.3636 & 54.2283 & 155.3547 & 68.8312 & 31.8182 & 50.3247 & 255.1079 \\ [0.5ex] \hline
				Media & 82.1875 & 32.5000 & 57.3438 & 487.6780 & 79.6818 & 37.2727 & 58.4772 & 158.7768 & 73.7662 & 35.0000 & 54.3831 & 329.6656 \\ [0.5ex] \hline
			\end{tabular}
		}
		\label{tablaMemetico10-1-BLXK1}
		\caption{Resultados AM(10,1.0) BLX con K=1}
	\end{table}

	\newpage

	\begin{table}[h!]
		\centering
		\resizebox{\textwidth}{!}{
			\begin{tabular}{| c | c | c | c | c | c | c | c | c | c | c | c | c |}
				\cline{2-13}
				\multicolumn{1}{c|}{} & \multicolumn{4}{|c|}{Ozone} & \multicolumn{4}{| c|}{Parkinsons} & \multicolumn{4}{|c|}{Spectf-Heart} \\ [0.5ex]
				\cline{2-13}
				\multicolumn{1}{c|}{} & \%\_clas & \%\_red & Agr. & T (seg) & \%\_clas & \%\_red & Agr. & T  (seg) & \%\_clas & \%\_red & Agr. & T  (seg) \\ [0.5ex] \hline
				Partición 1 & 71.8750 & 40.2778 & 56.0764 & 465.8056 & 71.0526 & 40.9091 & 55.9809 & 156.7125 & 72.0588 & 38.6364 & 55.3476 & 266.8972 \\ [0.5ex] \hline
				Partición 2 & 78.1250 & 34.7223 & 56.4236 & 472.3412 & 73.6842 & 40.9091 & 57.2967 & 150.1130 & 76.4706 & 38.6364 & 57.5535 & 340.3339 \\ [0.5ex] \hline
				Partición 3 & 75.0000 & 36.1111 & 55.5556 & 463.0502 & 94.7368 & 50.0000 & 72.3684 & 141.9956 & 70.5882 & 38.6364 & 54.6123 & 414.4833 \\ [0.5ex] \hline
				Partición 4 & 79.6875 & 31.9444 & 55.8160 & 472.0138 & 65.7895 & 50.0000 & 57.8947 & 155.8208 & 75.0000 & 38.6364 & 56.8182 & 379.9297 \\ [0.5ex] \hline
				Partición 5 & 81.2500 & 33.3333 & 57.2917 & 471.9598 & 74.4186 & 50.0000 & 62.2093 & 141.4247 & 67.5325 & 40.9091 & 54.2208 & 266.9989 \\ [0.5ex] \hline
				Media & 77.1875 & 35.2778 & 56.2326 & 469.0341 & 75.9364 & 46.3636 & 61.1499 & 149.2133 & 72.3300 & 39.0909 & 55.7105 & 333.7286 \\ [0.5ex] \hline
			\end{tabular}
		}
		\label{tablaMemetico10-01-BLXK1}
		\caption{Resultados AM(10,0.1) BLX con K=1}
	\end{table}

	\begin{table}[h!]
		\centering
		\resizebox{\textwidth}{!}{
			\begin{tabular}{| c | c | c | c | c | c | c | c | c | c | c | c | c |}
				\cline{2-13}
				\multicolumn{1}{c|}{} & \multicolumn{4}{|c|}{Ozone} & \multicolumn{4}{| c|}{Parkinsons} & \multicolumn{4}{|c|}{Spectf-Heart} \\ [0.5ex]
				\cline{2-13}
				\multicolumn{1}{c|}{} & \%\_clas & \%\_red & Agr. & T (seg) & \%\_clas & \%\_red & Agr. & T  (seg) & \%\_clas & \%\_red & Agr. & T  (seg) \\ [0.5ex] \hline
				Partición 1 & 73.4375 & 33.3333 & 53.3854 & 465.9320 & 73.6842 & 45.4545 & 59.5694 & 148.1477 & 72.0588 & 38.6364 & 55.3476 & 295.3414 \\ [0.5ex] \hline
				Partición 2 & 82.8125 & 34.7222 & 58.7674 & 472.6991 & 78.9474 & 45.4545 & 62.2010 & 154.1124 & 76.4706 & 38.6364 & 57.5535 & 370.5857 \\ [0.5ex] \hline
				Partición 3 & 68.7500 & 36.1111 & 52.4306 & 463.2197 & 86.8421 & 40.9091 & 63.8756 & 143.0956 & 69.1176 & 36.3636 & 52.7406 & 398.8947 \\ [0.5ex] \hline
				Partición 4 & 79.6875 & 33.3333 & 56.5104 & 472.5732 & 71.0526 & 45.4545 & 58.2536 & 142.6770 & 73.5294 & 38.6364 & 56.0829 & 342.1724 \\ [0.5ex] \hline
				Partición 5 & 82.8125 & 34.7222 & 58.7674 & 472.2727 & 72.0930 & 40.9091 & 56.5011 & 145.4911 & 71.4286 & 38.6364 & 55.0325 & 240.2233 \\ [0.5ex] \hline
				Media & 77.5000 & 34.4444 & 55.9722 & 469.3393 & 76.5239 & 43.6364 & 60.0801 & 146.7048 & 72.5210 & 38.1818 & 55.3514 & 329.4435 \\ [0.5ex] \hline
			\end{tabular}
		}
		\label{tablaMemetico10-01-mejores-BLXK1}
		\caption{Resultados AM(10,0.1,mejores) BLX con K=1}
	\end{table}

	\begin{table}[h!]
		\centering
		\resizebox{\textwidth}{!}{
			\begin{tabular}{| c | c | c | c | c | c | c | c | c | c | c | c | c |}
				\cline{2-13}
				\multicolumn{1}{c|}{} & \multicolumn{4}{|c|}{Ozone} & \multicolumn{4}{| c|}{Parkinsons} & \multicolumn{4}{|c|}{Spectf-Heart} \\ [0.5ex]
				\cline{2-13}
				\multicolumn{1}{c|}{} & \%\_clas & \%\_red & Agr. & T (seg) & \%\_clas & \%\_red & Agr. & T  (seg) & \%\_clas & \%\_red & Agr. & T  (seg) \\ [0.5ex] \hline
				Partición 1 & 76.5625 & 30.5556 & 53.5590 & 544.5574 & 86.8421 & 31.8182 & 59.3301 & 208.2288 & 83.8235 & 31.8182 & 57.8209 & 336.4153 \\ [0.5ex] \hline
				Partición 2 & 82.8125 & 33.3333 & 58.0729 & 552.3451 & 76.3158 & 36.3636 & 56.3397 & 216.6982 & 76.4706 & 31.8182 & 54.1444 & 426.7882 \\ [0.5ex] \hline
				Partición 3 & 71.8750 & 25.0000 & 48.4375 & 540.2087 & 92.1053 & 40.9091 & 66.5072 & 233.5908 & 76.4706 & 25.0000 & 50.7353 & 468.6593 \\ [0.5ex] \hline
				Partición 4 & 81.2500 & 30.5556 & 55.9028 & 552.1704 & 76.3158 & 40.9091 & 58.6124 & 257.1846 & 69.1176 & 31.8182 & 50.4679 & 464.2057 \\ [0.5ex] \hline
				Partición 5 & 79.6875 & 41.6667 & 60.6771 & 551.9782 & 67.4419 & 31.8182 & 49.6300 & 249.0803 & 71.4286 & 27.2727 & 49.3506 & 338.4131 \\ [0.5ex] \hline
				Media & 78.4375 & 32.2222 & 55.3299 & 548.2520 & 79.8042 & 36.3636 & 58.0839 & 232.9565 & 75.4622 & 29.5454 & 52.5038 & 406.8963 \\ [0.5ex] \hline
			\end{tabular}
		}
		\label{tablaMemetico10-1-CAK1}
		\caption{Resultados AM(10,1.0) CA con K=1}
	\end{table}

	\begin{table}[h!]
		\centering
		\resizebox{\textwidth}{!}{
			\begin{tabular}{| c | c | c | c | c | c | c | c | c | c | c | c | c |}
				\cline{2-13}
				\multicolumn{1}{c|}{} & \multicolumn{4}{|c|}{Ozone} & \multicolumn{4}{| c|}{Parkinsons} & \multicolumn{4}{|c|}{Spectf-Heart} \\ [0.5ex]
				\cline{2-13}
				\multicolumn{1}{c|}{} & \%\_clas & \%\_red & Agr. & T (seg) & \%\_clas & \%\_red & Agr. & T  (seg) & \%\_clas & \%\_red & Agr. & T  (seg) \\ [0.5ex] \hline
				Partición 1 & 76.5625 & 33.3333 & 54.9479 & 830.5894 & 65.7895 & 45.4545 & 55.6220 & 419.5533 & 73.5294 & 45.4545 & 59.4920 & 579.7760 \\ [0.5ex] \hline
				Partición 2 & 87.5000 & 38.8889 & 63.1944 & 838.5973 & 89.4737 & 45.4545 & 67.4641 & 420.6409 & 76.4706 & 40.9091 & 58.6898 & 694.4176 \\ [0.5ex] \hline
				Partición 3 & 79.6875 & 38.8889 & 59.2882 & 822.3652 & 78.9474 & 50.0000 & 64.4737 & 407.4625 & 76.4706 & 38.6364 & 57.5535 & 815.4603 \\ [0.5ex] \hline
				Partición 4 & 81.2500 & 40.2778 & 60.7639 & 838.2995 & 68.4211 & 40.9091 & 54.6651 & 416.2755 & 63.2353 & 36.3636 & 49.7995 & 763.9080 \\ [0.5ex] \hline
				Partición 5 & 78.1250 & 33.3333 & 55.7292 & 839.6178 & 76.7442 & 50.0000 & 63.3721 & 393.3524 & 71.4286 & 40.9091 & 56.1688 & 541.2438 \\ [0.5ex] \hline
				Media & 80.6250 & 36.9444 & 58.7847 & 833.8938 & 75.8752 & 46.3636 & 61.1194 & 411.4569 & 72.2269 & 40.4545 & 56.3407 & 678.9612 \\ [0.5ex] \hline
			\end{tabular}
		}
		\label{tablaMemetico10-01-CAK1}
		\caption{Resultados AM(10,0.1) CA con K=1}
	\end{table}

	\newpage

	\begin{table}[h!]
		\centering
		\resizebox{\textwidth}{!}{
			\begin{tabular}{| c | c | c | c | c | c | c | c | c | c | c | c | c |}
				\cline{2-13}
				\multicolumn{1}{c|}{} & \multicolumn{4}{|c|}{Ozone} & \multicolumn{4}{| c|}{Parkinsons} & \multicolumn{4}{|c|}{Spectf-Heart} \\ [0.5ex]
				\cline{2-13}
				\multicolumn{1}{c|}{} & \%\_clas & \%\_red & Agr. & T (seg) & \%\_clas & \%\_red & Agr. & T  (seg) & \%\_clas & \%\_red & Agr. & T  (seg) \\ [0.5ex] \hline
				Partición 1 & 71.8750 & 36.1111 & 53.9931 & 834.7623 & 86.8421 & 40.9091 & 63.8756 & 416.4395 & 73.5294 & 27.2727 & 50.4011 & 555.4199 \\ [0.5ex] \hline
				Partición 2 & 84.3750 & 38.8889 & 61.6319 & 837.4081 & 73.6842 & 31.8182 & 52.7512 & 407.1841 & 77.9412 & 29.5455 & 53.7433 & 689.7598 \\ [0.5ex] \hline
				Partición 3 & 75.0000 & 33.3333 & 54.1667 & 822.9041 & 100.0000 & 27.2727 & 63.6364 & 379.1863 & 66.1765 & 38.6364 & 52.4064 & 744.1559 \\ [0.5ex] \hline
				Partición 4 & 82.8125 & 33.3333 & 58.0729 & 837.5973 & 81.5789 & 40.9091 & 61.2440 & 348.5440 & 63.2353 & 34.0909 & 48.6631 & 697.9048 \\ [0.5ex] \hline
				Partición 5 & 81.2500 & 25.0000 & 53.1250 & 816.5085 & 76.7442 & 45.4545 & 61.0994 & 316.2853 & 70.1299 & 34.0909 & 52.1104 & 487.3722 \\ [0.5ex] \hline
				Media & 79.0625 & 33.3333 & 56.1979 & 829.8361 & 83.7699 & 37.2727 & 60.5213 & 373.5279 & 70.2024 & 32.7273 & 51.4649 & 634.9225 \\ [0.5ex] \hline
			\end{tabular}
		}
		\label{tablaMemetico10-01-mejores-CAK1}
		\caption{Resultados AM(10,0.1,mejores) CA con K=1}
	\end{table}

	\begin{table}[ht]
		\centering
		\resizebox{\textwidth}{!}{
			\begin{tabular}{| c | c | c | c | c | c | c | c | c | c | c | c | c |}
				\cline{2-13}
				\multicolumn{1}{c|}{} & \multicolumn{4}{|c|}{Ozone} & \multicolumn{4}{| c|}{Parkinsons} & \multicolumn{4}{|c|}{Spectf-Heart} \\ [0.5ex]
				\cline{2-13}
				\multicolumn{1}{c|}{} & \%\_clas & \%\_red & Agr. & T (seg) & \%\_clas & \%\_red & Agr. & T  (seg) & \%\_clas & \%\_red & Agr. & T  (seg) \\ [0.5ex] \hline
				Partición 1 &  &  &  &  &  &  &  &  &  &  &  &  \\ [0.5ex] \hline
				Partición 2 &  &  &  &  &  &  &  &  &  &  &  &  \\ [0.5ex] \hline
				Partición 3 &  &  &  &  &  &  &  &  &  &  &  &  \\ [0.5ex] \hline
				Partición 4 &  &  &  &  &  &  &  &  &  &  &  &  \\ [0.5ex] \hline
				Partición 5 &  &  &  &  &  &  &  &  &  &  &  &  \\ [0.5ex] \hline
				Media &  &  &  &  &  &  &  &  &  &  &  &  \\ [0.5ex] \hline
			\end{tabular}
		}
		\label{tablaES}
		\caption{Resultados ES}

	\end{table}
	
	\begin{table}[ht]
		\centering
		\resizebox{\textwidth}{!}{
			\begin{tabular}{| c | c | c | c | c | c | c | c | c | c | c | c | c |}
				\cline{2-13}
				\multicolumn{1}{c|}{} & \multicolumn{4}{|c|}{Ozone} & \multicolumn{4}{| c|}{Parkinsons} & \multicolumn{4}{|c|}{Spectf-Heart} \\ [0.5ex]
				\cline{2-13}
				\multicolumn{1}{c|}{} & \%\_clas & \%\_red & Agr. & T (seg) & \%\_clas & \%\_red & Agr. & T  (seg) & \%\_clas & \%\_red & Agr. & T  (seg) \\ [0.5ex] \hline
				Partición 1 &  &  &  &  &  &  &  &  &  &  &  &  \\ [0.5ex] \hline
				Partición 2 &  &  &  &  &  &  &  &  &  &  &  &  \\ [0.5ex] \hline
				Partición 3 &  &  &  &  &  &  &  &  &  &  &  &  \\ [0.5ex] \hline
				Partición 4 &  &  &  &  &  &  &  &  &  &  &  &  \\ [0.5ex] \hline
				Partición 5 &  &  &  &  &  &  &  &  &  &  &  &  \\ [0.5ex] \hline
				Media &  &  &  &  &  &  &  &  &  &  &  &  \\ [0.5ex] \hline
			\end{tabular}
		}
		\label{tablaILS}
		\caption{Resultados ILS}
		
	\end{table}
	
	\begin{table}[ht]
		\centering
		\resizebox{\textwidth}{!}{
			\begin{tabular}{| c | c | c | c | c | c | c | c | c | c | c | c | c |}
				\cline{2-13}
				\multicolumn{1}{c|}{} & \multicolumn{4}{|c|}{Ozone} & \multicolumn{4}{| c|}{Parkinsons} & \multicolumn{4}{|c|}{Spectf-Heart} \\ [0.5ex]
				\cline{2-13}
				\multicolumn{1}{c|}{} & \%\_clas & \%\_red & Agr. & T (seg) & \%\_clas & \%\_red & Agr. & T  (seg) & \%\_clas & \%\_red & Agr. & T  (seg) \\ [0.5ex] \hline
				Partición 1 &  &  &  &  &  &  &  &  &  &  &  &  \\ [0.5ex] \hline
				Partición 2 &  &  &  &  &  &  &  &  &  &  &  &  \\ [0.5ex] \hline
				Partición 3 &  &  &  &  &  &  &  &  &  &  &  &  \\ [0.5ex] \hline
				Partición 4 &  &  &  &  &  &  &  &  &  &  &  &  \\ [0.5ex] \hline
				Partición 5 &  &  &  &  &  &  &  &  &  &  &  &  \\ [0.5ex] \hline
				Media &  &  &  &  &  &  &  &  &  &  &  &  \\ [0.5ex] \hline
			\end{tabular}
		}
		\label{tablaDERand1}
		\caption{Resultados DE Rand1}
		
	\end{table}
	
	\begin{table}[ht]
		\centering
		\resizebox{\textwidth}{!}{
			\begin{tabular}{| c | c | c | c | c | c | c | c | c | c | c | c | c |}
				\cline{2-13}
				\multicolumn{1}{c|}{} & \multicolumn{4}{|c|}{Ozone} & \multicolumn{4}{| c|}{Parkinsons} & \multicolumn{4}{|c|}{Spectf-Heart} \\ [0.5ex]
				\cline{2-13}
				\multicolumn{1}{c|}{} & \%\_clas & \%\_red & Agr. & T (seg) & \%\_clas & \%\_red & Agr. & T  (seg) & \%\_clas & \%\_red & Agr. & T  (seg) \\ [0.5ex] \hline
				Partición 1 &  &  &  &  &  &  &  &  &  &  &  &  \\ [0.5ex] \hline
				Partición 2 &  &  &  &  &  &  &  &  &  &  &  &  \\ [0.5ex] \hline
				Partición 3 &  &  &  &  &  &  &  &  &  &  &  &  \\ [0.5ex] \hline
				Partición 4 &  &  &  &  &  &  &  &  &  &  &  &  \\ [0.5ex] \hline
				Partición 5 &  &  &  &  &  &  &  &  &  &  &  &  \\ [0.5ex] \hline
				Media &  &  &  &  &  &  &  &  &  &  &  &  \\ [0.5ex] \hline
			\end{tabular}
		}
		\label{tablaDECTB1}
		\caption{Resultados DE Current to Best 1}
		
	\end{table}

	\newpage

	\begin{table}[h!]
		\centering
		\resizebox{\textwidth}{!}{
			\begin{tabular}{| c | c | c | c | c | c | c | c | c | c | c | c | c |}
				\cline{2-13}
				\multicolumn{1}{c|}{} & \multicolumn{4}{|c|}{Ozone} & \multicolumn{4}{| c|}{Parkinsons} & \multicolumn{4}{|c|}{Spectf-Heart} \\ [0.5ex]
				\cline{2-13}
				\multicolumn{1}{c|}{} & \%\_clas & \%\_red & Agr. & T (seg) & \%\_clas & \%\_red & Agr. & T  (seg) & \%\_clas & \%\_red & Agr. & T  (seg) \\ [0.5ex] \hline
				1-NN & 79.0625 & 0.0000 & 39.5313 & 0.0096 & 80.6120 & 0.0000 & 40.3060 & 0.0034 & 68.5409 & 0.0000 & 34.2704 & 0.0078 \\ [0.5ex] \hline
				Relief & 75.9375 & 0.0000 & 37.9688 & 1.8086 & 72.8397 & 0.0000 & 36.4198 & 0.2812 & 55.3858 & 0.0000 & 27.6929 & 0.9882 \\ [0.5ex] \hline
				BL & 76.875 & 25.2778 & 51.0764 & 90.1781 & 82.7173 & 20.9091 & 51.8132 & 6.5444 & 68.3499 & 25.9091 & 47.1295 & 36.6666 \\ [0.5ex] \hline
				AGE-BLX & 80.6250 & 52.5000 & 66.5625 & 485.5372 & 75.4100 & 70.0000 & 72.7050 & 171.4736 & 70.6532 & 60.0000 & 65.3266 & 316.7908 \\ [0.5ex] \hline
				AGE-CA & 80.6250 & 69.7222 & 75.1736 & 473.6093 & 72.7173 & 71.8182 & 72.2677 & 192.4899 & 72.1581 & 75.4545 & 73.8063 & 321.5867 \\ [0.5ex] \hline
				AGG-BLX & 76.5625 & 42.7778 & 59.6701 & 653.2276 & 78.8127 & 49.0909 & 63.9518 & 269.2324 & 68.2811 & 48.6364 & 58.4587 & 461.5902 \\ [0.5ex] \hline
				AGG-CA & 80.0000 & 46.3889 & 63.1944 & 1301.5695 & 83.7087 & 36.3636 & 60.0362 & 479.5455 & 69.9083 & 51.3636 & 60.6360 & 939.6510 \\ [0.5ex] \hline
				AM(10,1) BLX & 82.1875 & 32.5000 & 57.3438 & 487.6780 & 79.6818 & 37.2727 & 58.4772 & 158.7768 & 73.7662 & 35.0000 & 54.3831 & 329.6656 \\ [0.5ex] \hline
				AM(10,0.1) BLX & 77.1875 & 35.2778 & 56.2326 & 469.0341 & 75.9364 & 46.3636 & 61.1499 & 149.2133 & 72.3300 & 39.0909 & 55.7105 & 333.7286 \\ [0.5ex] \hline
				AM(10,0.1,mejores) BLX & 77.5000 & 34.4444 & 55.9722 & 469.3393 & 76.5239 & 43.6364 & 60.0801 & 146.7048 & 72.5210 & 38.1818 & 55.3514 & 329.4435 \\ [0.5ex] \hline
				AM(10,1) CA & 78.4375 & 32.2222 & 55.3299 & 548.2520 & 79.8042 & 36.3636 & 58.0839 & 232.9565 & 75.4622 & 29.5454 & 52.5038 & 406.8963 \\ [0.5ex] \hline
				AM(10,0.1) CA & 80.6250 & 36.9444 & 58.7847 & 833.8938 & 75.8752 & 46.3636 & 61.1194 & 411.4569 & 72.2269 & 40.4545 & 56.3407 & 678.9612 \\ [0.5ex] \hline
				AM(10,0.1,mejores) CA & 79.0625 & 33.3333 & 56.1979 & 829.8361 & 83.7699 & 37.2727 & 60.5213 & 373.5279 & 70.2024 & 32.7273 & 51.4649 & 634.9225 \\ [0.5ex] \hline
				ES &  &  &  &  &  &  &  &  &  &  &  &  \\ [0.5ex] \hline
				ILS &  &  &  &  &  &  &  &  &  &  &  &  \\ [0.5ex] \hline
				DE Rand1 &  &  &  &  &  &  &  &  &  &  &  &  \\ [0.5ex] \hline
				DE CTB1 &  &  &  &  &  &  &  &  &  &  &  &  \\ [0.5ex] \hline
			\end{tabular}
		}
		\label{tablaGlobalK1}
		\caption{Resultados globales con K=1}
	\end{table}


	\newpage

	\subsection{Análisis de los datos}


\end{document}

\documentclass[12pt,a4paper]{article}
\usepackage[utf8]{inputenc}
\usepackage[spanish]{babel}
\usepackage{amsmath}
\usepackage{amsfonts}
\usepackage{amssymb}
\usepackage{graphicx}
\usepackage[left=2cm,right=2cm,top=2cm,bottom=2cm]{geometry}

%--------------------------------------------------------%
%         Paquetes usados sólo para esta entrega         %
%--------------------------------------------------------%

\usepackage{hyperref}
\usepackage{algorithm}
\usepackage{algorithmic}



\author{Ignacio Aguilera Martos \\
	DNI: 77448262V       e-mail: nacheteam@correo.ugr.es \\
	Grupo de prácticas 1 Lunes 17:30-19:30}
\title{Práctica 1 Búsqueda Local APC \\ Metaheurísticas \\ Algoritmos KNN, Relief y Búsqueda Local}
\date{Curso 2017-2018}

%Quita la sangría
\setlength{\parindent}{0cm}


\begin{document}
	\maketitle

	\tableofcontents

	\newpage

	%p 56

%	\framebox[16cm][c]{\LaTeX}

	\section{Introducción del problema}
	\label{sec:introProblema}
	
	Para el problema de clasificación partimos de un conjunto de datos dado por una serie de tuplas que contienen los valores de atributos para cada instancia. Esto es una n-tupla de valores reales en nuestro caso. 

	El objetivo del problema es obtener un vector de pesos que asocia un valor en el intervalo $[0,1]$ indicativo de la relevancia de ese atributo. Esta relevancia va referida a lo importante que es en nuestro algoritmo clasificador ese atributo a la hora de computar la distancia entre elementos.
	
	Resumiendo lo que tenemos es un algoritmo clasificador que utiliza el vector de pesos calculado para predecir la clase a la que pertenece una instancia dada. Este algoritmo clasificador es el KNN con k=1. Lo que hace es calcular según la distancia euclídea (o cualquier otra) la tupla más cercana a la que queremos clasificar ponderando cada atributo con el correspondiente peso del vector, es decir, la distancia entre dos elementos sería:
	$$d(e,f) = \sqrt{\sum_{i=0}^{n}w_i * (e_{i} - f_{i})}$$
	Donde e y f son instancias del conjunto de datos, w el vector de pesos y n la longitud de e y f que es la misma.
	
	La calificación que se le asigna al vector w depende de dos cosas: la tasa de aciertos y la simplicidad.
	
	La tasa de aciertos se mide contando el número de aciertos al emplear el clasificador descrito y la simplicidad se mide como el número de elementos del vector de pesos que son menores que 0.2, ya que estos pesos no son empleados por el clasificador, o lo que es lo mismo, son sustituidos por cero. Por lo tanto las calificaciones siguen las fórmulas:
	$$Tasa\_acierto = 100\cdot \frac{nº aciertos}{nº datos} \ , \ Tasa\_simplicidad = 100\cdot \frac{nº valores de w < 0.2}{nº de atributos}$$
	$$Tasa\_agregada = \frac{1}{2}\cdot Tasa\_acierto + \frac{1}{2}\cdot Tasa\_simplicidad$$
	Cabe destacar que todas las tasas están expresadas en porcentajes, por lo tanto cuanto más cercano sea el valor a 100 mejor es la calificación.
	
	De esta forma a través del algoritmo que obtiene el vector de pesos para el conjunto de datos dado y el clasificador obtenemos un programa que clasifica de forma automática las nuevas instancias de datos que se introduzcan.
	
	\newpage

	\section{Introducción de la práctica}
	\label{sec:introPractica}
	
	En esta práctica he analizado el comportamiento de los algoritmos KNN con k=1, el algoritmo greedy Relief y una implementación del algoritmo de búsqueda local para el problema de obtención de un vector de pesos para clasificar un conjunto de datos. Así mismo he realizado la implementación del algoritmo KNN con k variable para poder estudiar si varían los resultados al aumentar el valor de K o por contra obtenemos demasiado ajuste.
	
	Para empezar al leer los ficheros de datos dados para el problema me he dado cuenta de que tenemos tuplas repetidas, cosa que he tenido en cuenta para no usarlas en la clasificación, ya que siempre obtendríamos distancia 0 para dicha tupla. Para ello en vez de comprobar el índice dentro del vector he comprobado si las tuplas son iguales para no usarlas.
	
	Así mismo he implementado diferentes distancias a parte de la euclídea para comprobar si los resultados son mejores o peores en función de la distancia para cada conjunto de datos.
	
	Para terminar, antes de analizar los datos, se debe considerar que los datos han sido redondeados a 4 decimales para no obtener tablas excesivamente largas. Si se desea obtener los datos completos se puede ejecutar el programa como se describe en la sección \hyperref[sec:programa]{\ref{sec:programa}}.


	\section{Descripción común a todos los algoritmos}
	Los algoritmos empleados han sido el KNN, el algoritmo greedy Relief y la metaheurística de búsqueda local. 
	
	Estos algoritmos comparten ciertos métodos y operadores que pasaré a explicar en esta sección.
	
	Para empezar se debe destacar que la representación escogida para las soluciones es un vector de números reales, es decir, si n es el número de características:
	$$w\in \mathbb{R}^n \ t.q. \ \forall i \ con \ 0\leq i < n \ se \ tiene \ w_i \in [0,1]$$
	O lo que es lo mismo, un vector de tamaño n con todas las posiciones rellenas con números del intervalo [0,1]. 
	
	A estos números me referiré como pesos asociados a las características, ya que lo que nos indican es el grado de importancia de dicha característica a la hora de clasificar los datos, siendo 1 el máximo de relevancia y 0 el mínimo.
	
	Así mismo cabe destacar que nuestra intención en este problema es obtener una buena calificación de dicho vector de pesos. Esto lo medimos mediante las tasas de acierto y simplicidad que se definen como:
	$$Tasa\_acierto = 100\cdot \frac{nº aciertos}{nº datos} \ , \ Tasa\_simplicidad = 100\cdot \frac{nº valores de w < 0.2}{nº de atributos}$$
	$$Tasa\_agregada = \frac{1}{2}\cdot Tasa\_acierto + \frac{1}{2}\cdot Tasa\_simplicidad$$
	
	La tasa de aciertos lo que nos mide es en un porcentaje cuántas instancias hemos clasificado correctamente mediante el algoritmo KNN usando el vector de pesos w.
	
	La tasa de simplicidad nos mide cuántos de los valores que tiene el vector de pesos son menores que 0.2. Esto se hace ya que como imposición del problema tenemos que si alguna de los pesos es menor que 0.2 no debemos usarlo, o lo que es lo mismo, debemos sustituirlo por un 0 en la función de la distancia que luego describiré. Midiendo esto obtenemos un dato de cuanto sobreajuste ha tenido nuestro algoritmo a la hora de obtener el vector de pesos. Cuantas menos características necesitemos para discernir la clase a la que pertenece una instancia de los datos, más simple será clasificar dicha instancia. Se expresa en porcentaje indicando 0 como ninguna simplicidad y 100 como la máxima simplicidad.
	
	De esta forma combinando ambas tasas obtenemos la tasa agregada que nos hace la media entre ambas tasas, de forma que le asignamos la misma importancia a acertar en la clasificación de las instancias y a la simplicidad en la solución. Cabe destacar que es imposible obtener una tasa de un 100\% a no ser que los datos se compongan únicamente de un punto ya que ello implicaría que la simplicidad ha de ser un 100\% (todos las posiciones del vector menores que 0.2) y por tanto la distancia sería 0 en todos los casos. De esta forma aspiraremos a una calificación lo mas alta posible pero teniendo en cuenta las restricciones de la función objetivo construida.
	
	Las funciones y operadores de uso común los he agrupado en un fichero llamado auxiliar.py. Este fichero contiene las funciones de lectura de datos, distancias, una función que devuelve el elemento más común de una lista, la norma euclídea y una función para dividir los datos en el número de particiones que queramos manteniendo el porcentaje de elementos de cada clase que había en el conjunto de datos original.
	
	\subsection{Función de lectura de datos}
	
	La función de lectura de datos recibe la ruta del fichero arff y lee el contenido del mismo dando como resultado una lista de listas en la que cada una de ellas es una tupla o instancia de los datos.
	
	El pseudocódigo de la función es:
	
	\begin{algorithm}
		\caption{lecturaDatos(nombre\_fich)}
		\label{algoritmoLecturaDatos}
		\begin{algorithmic}
			\STATE data $\leftarrow$ [ ]
			\FOR{linea de nombre\_fich}
				\IF{se ha leído @data}
					\STATE data $\leftarrow$ [data,linea]
				\ENDIF
			\ENDFOR
			\RETURN data
		\end{algorithmic}
	\end{algorithm}
	
	Para esta implementación en concreto nos apoyamos en que Python tiene polimorfismo para todos los tipos de datos sin necesidad de declarar las variables, de forma que no nos importa que los datos sean numéricos o de tipo string.
	
	
	

	\section{KNN}
	\label{sec:knn}

	\begin{table}[ht]
	\centering
	\resizebox{\textwidth}{!}{
		\begin{tabular}{| c | c | c | c | c | c | c | c | c | c | c | c | c |}
		\cline{2-13}
		\multicolumn{1}{c|}{} & \multicolumn{4}{|c|}{Ozone} & \multicolumn{4}{| c|}{Parkinsons} & \multicolumn{4}{|c|}{Spectf-Heart} \\ [0.5ex]
		\cline{2-13}
		\multicolumn{1}{c|}{} & \%\_clas & \%\_red & Agr. & T (seg) & \%\_clas & \%\_red & Agr. & T  (seg) & \%\_clas & \%\_red & Agr. & T  (seg) \\ [0.5ex] \hline
		Partición 1 & 60.9375 & 0.0000 & 30.4688 & 0.1136 & 100.0000 & 0.0000 & 50.0000 & 0.0106 & 69.1176 & 0.0000 & 34.5588 & 0.0613 \\ [0.5ex] \hline
		Partición 2 & 71.8750 & 0.0000 & 35.9375 & 0.1116 & 97.3684 & 0.0000 & 48.6842 & 0.0103 & 67.6470 & 0.0000 & 33.8236 & 0.0592 \\ [0.5ex] \hline
		Partición 3 & 67.1875 & 0.0000 & 33.5938 & 0.1118 & 97.3684 & 0.0000 & 48.6842 & 0.0107 & 75.0000 & 0.0000 & 37.5000 & 0.0604 \\ [0.5ex] \hline
		Partición 4 & 60.9375 & 0.0000 & 30.4688 & 0.1123 & 81.5789 & 0.0000 & 40.7895 & 0.0102 & 66.1765 & 0.0000 & 33.0882 & 0.0609 \\ [0.5ex] \hline
		Partición 5 & 64.0625 & 0.0000 & 32.0313 & 0.1127 & 76.7442 & 0.0000 & 38.3721 & 0.0130 & 61.0390 & 0.0000 & 30.5195 & 0.0767 \\ [0.5ex] \hline
		Media & 65.0000 & 0.0000 & 32.5000 & 0.1122 & 90.6120 & 0.0000 & 45.3060 & 0.0144 & 67.7960 & 0.0000 & 33.8980 & 0.0836 \\ [0.5ex] \hline
		\end{tabular}
	}
	\label{tabla1NN}
	\caption{Resultados 1NN}
	\end{table}
	
	\begin{table}[ht]
	\centering
	\resizebox{\textwidth}{!}{
		\begin{tabular}{| c | c | c | c | c | c | c | c | c | c | c | c | c |}
		\cline{2-13}
		\multicolumn{1}{c|}{} & \multicolumn{4}{|c|}{Ozone} & \multicolumn{4}{| c|}{Parkinsons} & \multicolumn{4}{|c|}{Spectf-Heart} \\ [0.5ex]
		\cline{2-13}
		\multicolumn{1}{c|}{} & \%\_clas & \%\_red & Agr. & T (seg) & \%\_clas & \%\_red & Agr. & T  (seg) & \%\_clas & \%\_red & Agr. & T  (seg) \\ [0.5ex] \hline
		Partición 1 & 67.1875 & 0.0000 & 33.5938 & 0.1127 & 100.0000 & 0.0000 & 50.0000 & 0.0113 & 72.0588 & 0.0000 & 36.0294 & 0.0641 \\ [0.5ex] \hline
		Partición 2 & 76.5625 & 0.0000 & 38.2813 & 0.1115 & 92.1053 & 0.0000 & 46.0526 & 0.0108 & 70.5882 & 0.0000 & 35.2941 & 0.0619 \\ [0.5ex] \hline
		Partición 3 & 62.5000 & 0.0000 & 31.2500 & 0.1123 & 94.7368 & 0.0000 & 47.3684 & 0.0119 & 79.4118 & 0.0000 & 39.7059 & 0.0607 \\ [0.5ex] \hline
		Partición 4 & 68.7500 & 0.0000 & 34.3750 & 0.1115 & 76.3158 & 0.0000 & 38.1579 & 0.0107 & 67.6471 & 0.0000 & 33.8235 & 0.0622 \\ [0.5ex] \hline
		Partición 5 & 76.5625 & 0.0000 & 38.2813 & 0.1119 & 76.7442 & 0.0000 & 38.3721 & 0.0135 & 71.4286 & 0.0000 & 35.7243 & 0.0773 \\ [0.5ex] \hline
		Media & 70.3125 & 0.0000 & 35.1563 & 0.1120 & 87.9804 & 0.0000 & 43.9902 & 0.0116 & 72.2269 & 0.0000 & 36.1134 & 0.0653 \\ [0.5ex] \hline
		\end{tabular}
	}
	\label{tabla3NN}
	\caption{Resultados 3NN}
	\end{table}
	
	\begin{table}[ht]
		\centering
		\resizebox{\textwidth}{!}{
			\begin{tabular}{| c | c | c | c | c | c | c | c | c | c | c | c | c |}
				\cline{2-13}
				\multicolumn{1}{c|}{} & \multicolumn{4}{|c|}{Ozone} & \multicolumn{4}{| c|}{Parkinsons} & \multicolumn{4}{|c|}{Spectf-Heart} \\ [0.5ex]
				\cline{2-13}
				\multicolumn{1}{c|}{} & \%\_clas & \%\_red & Agr. & T (seg) & \%\_clas & \%\_red & Agr. & T  (seg) & \%\_clas & \%\_red & Agr. & T  (seg) \\ [0.5ex] \hline
				Partición 1 & 70.3125 & 0.0000 & 35.1563 & 0.1205 & 100.0000 & 0.0000 & 50.0000 & 0.0122 & 73.5294 & 0.0000 & 36.7647 & 0.0656 \\ [0.5ex] \hline
				Partición 2 & 78.1250 & 0.0000 & 39.0625 & 0.1207 & 86.8421 & 0.0000 & 43.4211 & 0.0123 & 75.0000 & 0.0000 & 37.5000 & 0.0625 \\ [0.5ex] \hline
				Partición 3 & 59.3750 & 0.0000 & 29.6875 & 0.1194 & 97.3684 & 0.0000 & 48.6842 & 0.0122 & 80.8824 & 0.0000 & 40.4412 & 0.0625 \\ [0.5ex] \hline
				Partición 4 & 68.7500 & 0.0000 & 34.3750 & 0.1192 & 76.3158 & 0.0000 & 38.1579 & 0.0121 & 70.5882 & 0.0000 & 35.2941 & 0.0653 \\ [0.5ex] \hline
				Partición 5 & 79.6875 & 0.0000 & 39.8438 & 0.1182 & 69.7674 & 0.0000 & 34.8837 & 0.0156 & 74.0260 & 0.00000 & 37.0130 & 0.0811 \\ [0.5ex] \hline
				Media & 71.2500 & 0.0000 & 35.6250 & 0.1196 & 86.0588 & 0.0000 & 43.0294 & 0.0129 & 74.8052 & 0.0000 & 37.4026 & 0.0674 \\ [0.5ex] \hline
			\end{tabular}
		}
		\label{tabla5NN}
		\caption{Resultados 5NN}
	\end{table}



	\section{Relief}
	\label{sec:relief}

	\begin{table}[ht]
		\centering
		\resizebox{\textwidth}{!}{
			\begin{tabular}{| c | c | c | c | c | c | c | c | c | c | c | c | c |}
				\cline{2-13}
				\multicolumn{1}{c|}{} & \multicolumn{4}{|c|}{Ozone} & \multicolumn{4}{| c|}{Parkinsons} & \multicolumn{4}{|c|}{Spectf-Heart} \\ [0.5ex]
				\cline{2-13}
				\multicolumn{1}{c|}{} & \%\_clas & \%\_red & Agr. & T (seg) & \%\_clas & \%\_red & Agr. & T  (seg) & \%\_clas & \%\_red & Agr. & T  (seg) \\ [0.5ex] \hline
				Partición 1 & 67.8431 & 94.5205 & 81.1818 & 0.0612 & 82.8025 & 86.9565 & 84.8795 & 0.0078 & 81.6143 & 48.8889 & 65.2516 & 0.0439 \\ [0.5ex] \hline
				Partición 2 & 68.7500 & 63.0137 & 65.8818 & 0.0642 & 81.5287 & 91.3043 & 86.4165 & 0.0081 & 90.4580 & 80.0000 & 85.2290 & 0.0435 \\ [0.5ex] \hline
				Partición 3 & 68.5039 & 82.1918 & 75.3479 & 0.0613 & 83.4395 & 86.9565 & 85.1980 & 0.0078 & 93.5018 & 51.1111 & 72.3065 & 0.0439 \\ [0.5ex] \hline
				Partición 4 & 61.7188 & 94.5205 & 78.1196 & 0.0657 & 82.8025 & 95.6522 & 89.2274 & 0.0079 & 89.0152 & 62.2222 & 75.6187 & 0.0436 \\ [0.5ex] \hline
				Partición 5 & 57.4219 & 95.8904 & 76.6561 & 0.0638 & 76.3158 & 86.9565 & 81.6362 & 0.0104 & 80.1932 & 42.2222 & 61.2077 & 0.0557 \\ [0.5ex] \hline
				Media & 64.8475 & 86.0274 & 75.4375 & 0.0607 & 81.3778 & 89.5652 & 85.4715 & 0.0101 & 86.9565 & 56.8889 & 71.9227 & 0.0568 \\ [0.5ex] \hline
			\end{tabular}
		}
		\label{tablaReliefK1}
		\caption{Resultados Relief con K=1}
	\end{table}
	
	\begin{table}[ht]
		\centering
		\resizebox{\textwidth}{!}{
			\begin{tabular}{| c | c | c | c | c | c | c | c | c | c | c | c | c |}
				\cline{2-13}
				\multicolumn{1}{c|}{} & \multicolumn{4}{|c|}{Ozone} & \multicolumn{4}{| c|}{Parkinsons} & \multicolumn{4}{|c|}{Spectf-Heart} \\ [0.5ex]
				\cline{2-13}
				\multicolumn{1}{c|}{} & \%\_clas & \%\_red & Agr. & T (seg) & \%\_clas & \%\_red & Agr. & T  (seg) & \%\_clas & \%\_red & Agr. & T  (seg) \\ [0.5ex] \hline
				Partición 1 & 65.8824 & 94.5205 & 80.2015 & 0.0607 & 84.7134 & 86.9565 & 85.8349 & 0.0078 & 69.9552 & 48.8889 & 59.4220 & 0.0457 \\ [0.5ex] \hline
				Partición 2 & 75.3906 & 63.0137 & 69.2022 & 0.0613 & 80.2548 & 91.3043 & 85.7796 & 0.0078 & 73.2824 & 80.0000 & 76.6412 & 0.0437 \\ [0.5ex] \hline
				Partición 3 & 74.0157 & 82.1918 & 78.1038 & 0.0626 & 82.8025 & 86.9565 & 84.8795 & 0.0086 & 76.1733 & 51.1111 & 63.6422 & 0.0453 \\ [0.5ex] \hline
				Partición 4 & 57.0313 & 94.5205 & 75.7759 & 0.0618 & 83.4395 & 95.6522 & 89.5458 & 0.0078 & 81.4394 & 62.2222 & 71.8308 & 0.0436 \\ [0.5ex] \hline
				Partición 5 & 63.6719 & 95.8904 & 79.7811 & 0.0599 & 85.5263 & 86.9565 & 86.2414 & 0.0099 & 73.9130 & 42.2222 & 58.0676 & 0.0571 \\ [0.5ex] \hline
				Media & 67.1984 & 86.0274 & 76.6129 & 0.0613 & 83.3473 & 89.5652 & 86.4563 & 0.0084 & 74.9527 & 56.8889 & 65.9208 & 0.0471 \\ [0.5ex] \hline
			\end{tabular}
		}
		\label{tablaReliefK3}
		\caption{Resultados Relief con K=3}
	\end{table}
	
	\begin{table}[ht]
		\centering
		\resizebox{\textwidth}{!}{
			\begin{tabular}{| c | c | c | c | c | c | c | c | c | c | c | c | c |}
				\cline{2-13}
				\multicolumn{1}{c|}{} & \multicolumn{4}{|c|}{Ozone} & \multicolumn{4}{| c|}{Parkinsons} & \multicolumn{4}{|c|}{Spectf-Heart} \\ [0.5ex]
				\cline{2-13}
				\multicolumn{1}{c|}{} & \%\_clas & \%\_red & Agr. & T (seg) & \%\_clas & \%\_red & Agr. & T  (seg) & \%\_clas & \%\_red & Agr. & T  (seg) \\ [0.5ex] \hline
				Partición 1 & 65.0980 & 94.5205 & 78.8093 & 0.0637 & 80.8917 & 86.9565 & 83.9241 & 0.0086 & 68.6099 & 48.8889 & 58.7494 & 0.0443 \\ [0.5ex] \hline
				Partición 2 & 70.3125 & 63.0137 & 66.6631 & 0.0614 & 79.6178 & 91.3043 & 85.4611 & 0.0086 & 74.4275 & 80.0000 & 77.2137 & 0.0470 \\ [0.5ex] \hline
				Partición 3 & 72.4409 & 82.1918 & 77.3164 & 0.0631 & 82.1656 & 86.9565 & 84.5611 & 0.0081 & 83.0325 & 51.1111 & 67.0718 & 0.0510 \\ [0.5ex] \hline
				Partición 4 & 56.6406 & 94.5205 & 75.5806 & 0.0625 & 83.4395 & 95.6522 & 89.5458 & 0.0088 & 81.4394 & 62.2222 & 71.8308 & 0.0466 \\ [0.5ex] \hline
				Partición 5 & 66.0156 & 95.8904 & 80.9530 & 0.0649 & 86.1842 & 86.9565 & 86.5704 & 0.0108 & 76.3285 & 42.2222 & 59.2754 & 0.0614 \\ [0.5ex] \hline
				Media & 66.1015 & 86.0274 & 76.0645 & 0.0631 & 82.4598 & 89.5652 & 86.0125 & 0.0090 & 76.7675 & 56.8889 & 66.8282 & 0.0501 \\ [0.5ex] \hline
			\end{tabular}
		}
		\label{tablaReliefK5}
		\caption{Resultados Relief con K=5}
	\end{table}

	\section{Búsqueda Local}
	\label{sec:bl}

	\begin{table}[ht]
		\centering
		\resizebox{\textwidth}{!}{
			\begin{tabular}{| c | c | c | c | c | c | c | c | c | c | c | c | c |}
				\cline{2-13}
				\multicolumn{1}{c|}{} & \multicolumn{4}{|c|}{Ozone} & \multicolumn{4}{| c|}{Parkinsons} & \multicolumn{4}{|c|}{Spectf-Heart} \\ [0.5ex]
				\cline{2-13}
				\multicolumn{1}{c|}{} & \%\_clas & \%\_red & Agr. & T (seg) & \%\_clas & \%\_red & Agr. & T  (seg) & \%\_clas & \%\_red & Agr. & T  (seg) \\ [0.5ex] \hline
				Partición 1 & 74.1176 & 53.4247 & 63.7712 & 34.5843 & 82.1656 & 43.4783 & 62.8219 & 0.5610 & 81.1659 & 35.5556 & 58.3607 & 5.3001 \\ [0.5ex] \hline
				Partición 2 & 67.1875 & 35.6164 & 51.4020 & 6.4653 & 82.8025 & 43.4783 & 63.1404 & 0.5545 & 87.4046 & 31.1111 & 59.2579 & 3.5880 \\ [0.5ex] \hline
				Partición 3 & 66.5354 & 53.4247 & 59.9800 & 13.2504 & 92.3567 & 39.1304 & 65.7436 & 0.6979 & 93.8629 & 40.0000 & 66.9314 & 4.7296 \\ [0.5ex] \hline
				Partición 4 & 71.0938 & 47.9452 & 59.5195 & 23.0546 & 89.1720 & 21.7391 & 55.4556 & 0.2889 & 87.5000 & 35.5556 & 61.5278 & 9.3608 \\ [0.5ex] \hline
				Partición 5 & 69.1406 & 43.8356 & 56.4881 & 10.8910 & 92.1053 & 43.4783 & 67.7917 & 0.3560 & 76.8116 & 51.1111 & 63.9613 & 11.3526 \\ [0.5ex] \hline
				Media & 69.6150 & 46.8493 & 58.2321 & 17.1638 & 87.7204 & 38.2609 & 62.9906 & 0.5722 & 85.3490 & 38.6667 & 62.0078 & 7.6077 \\ [0.5ex] \hline
			\end{tabular}
		}
		\label{tablaBLK1}
		\caption{Resultados Búsqueda Local con K=1}
	\end{table}
	
	\begin{table}[ht]
		\centering
		\resizebox{\textwidth}{!}{
			\begin{tabular}{| c | c | c | c | c | c | c | c | c | c | c | c | c |}
				\cline{2-13}
				\multicolumn{1}{c|}{} & \multicolumn{4}{|c|}{Ozone} & \multicolumn{4}{| c|}{Parkinsons} & \multicolumn{4}{|c|}{Spectf-Heart} \\ [0.5ex]
				\cline{2-13}
				\multicolumn{1}{c|}{} & \%\_clas & \%\_red & Agr. & T (seg) & \%\_clas & \%\_red & Agr. & T  (seg) & \%\_clas & \%\_red & Agr. & T  (seg) \\ [0.5ex] \hline
				Partición 1 & 69.8039 & 43.8356 & 56.8198 & 12.9352 & 88.5350 & 39.1304 & 63.8327 & 0.3989 & 75.7848 & 28.8889 & 52.3368 & 4.6527 \\ [0.5ex] \hline
				Partición 2 & 69.5313 & 50.6849 & 60.1081 & 18.8173 & 81.5287 & 26.0869 & 53.8078 & 0.2219 & 78.2443 & 44.4444 & 61.3444 & 4.9285 \\ [0.5ex] \hline
				Partición 3 & 70.4724 & 43.8356 & 57.1540 & 13.0915 & 82.1656 & 43.4783 & 62.8219 & 1.0002 & 71.4801 & 53.3333 & 62.4067 & 7.6825 \\ [0.5ex] \hline
				Partición 4 & 67.1875 & 41.0959 & 54.1417 & 17.3212 & 89.1720 & 34.7826 & 61.9773 & 0.5576 & 74.6212 & 37.7778 & 56.1995 & 7.1972 \\ [0.5ex] \hline
				Partición 5 & 67.5781 & 46.5753 & 57.0767 & 22.6021 & 87.5000 & 47.8261 & 67.6630 & 0.7952 & 77.7778 & 37.7778 & 57.7778 & 9.2464 \\ [0.5ex] \hline
				Media & 68.9146 & 45.2055 & 57.0601 & 16.9535 & 85.7803 & 38.2609 & 62.0206 & 0.5948 & 75.5816 & 40.4444 & 58.0130 & 6.7415 \\ [0.5ex] \hline
			\end{tabular}
		}
		\label{tablaBLK3}
		\caption{Resultados Búsqueda Local con K=3}
	\end{table}
	
	\begin{table}[ht]
		\centering
		\resizebox{\textwidth}{!}{
			\begin{tabular}{| c | c | c | c | c | c | c | c | c | c | c | c | c |}
				\cline{2-13}
				\multicolumn{1}{c|}{} & \multicolumn{4}{|c|}{Ozone} & \multicolumn{4}{| c|}{Parkinsons} & \multicolumn{4}{|c|}{Spectf-Heart} \\ [0.5ex]
				\cline{2-13}
				\multicolumn{1}{c|}{} & \%\_clas & \%\_red & Agr. & T (seg) & \%\_clas & \%\_red & Agr. & T  (seg) & \%\_clas & \%\_red & Agr. & T  (seg) \\ [0.5ex] \hline
				Partición 1 & 66.6667 & 50.6849 & 58.6758 & 16.2974 & 83.4395 & 34.7826 & 59.1110 & 0.8260 & 74.8879 & 40.0000 & 57.4439 & 11.5613 \\ [0.5ex] \hline
				Partición 2 & 64.0625 & 32.8767 & 48.4696 & 9.4039 & 83.4395 & 21.7391 & 52.5893 & 0.3869 & 74.4275 & 40.0000 & 57.2137 & 9.4207 \\ [0.5ex] \hline
				Partición 3 & 74.8031 & 57.5342 & 66.1687 & 24.0036 & 84.7134 & 47.8261 & 66.2697 & 0.8540 & 83.7545 & 44.4444 & 64.0995 & 9.1497 \\ [0.5ex] \hline
				Partición 4 & 76.1719 & 60.2740 & 68.2229 & 34.7944 & 88.5350 & 52.1740 & 70.3545 & 0.7861 & 78.0303 & 57.7778 & 67.9040 & 9.0839 \\ [0.5ex] \hline
				Partición 5 & 68.7500 & 47.9452 & 58.3476 & 20.8456 & 96.0526 & 34.7826 & 65.4176 & 0.5576 & 78.2609 & 37.7778 & 58.0193 & 4.626 \\ [0.5ex] \hline
				Media & 70.0908 & 49.8630 & 59.9769 & 21.0690 & 87.2360 & 38.2609 & 62.7484 & 0.6821 & 77.8722 & 44.0000 & 60.9361 & 8.8356 \\ [0.5ex] \hline
			\end{tabular}
		}
		\label{tablaBLK5}
		\caption{Resultados Búsqueda Local con K=5}
	\end{table}
	
	\section{Ejecución del programa y explicación}
	\label{sec:programa}

\end{document}

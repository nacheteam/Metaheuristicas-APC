\documentclass[12pt,a4paper]{article}
\usepackage[utf8]{inputenc}
\usepackage[spanish]{babel}
\usepackage{amsmath}
\usepackage{amsfonts}
\usepackage{amssymb}
\usepackage{graphicx}
\usepackage[left=2cm,right=2cm,top=2cm,bottom=2cm]{geometry}

%--------------------------------------------------------%
%         Paquetes usados sólo para esta entrega         %
%--------------------------------------------------------%

\usepackage{hyperref}
\usepackage{algorithm}
\usepackage{algorithmic}



\author{Ignacio Aguilera Martos \\
	DNI: 77448262V       e-mail: nacheteam@correo.ugr.es \\
	Grupo de prácticas 1 Lunes 17:30-19:30}
\title{Práctica Algoritmos Genéticos y Meméticos APC \\ Metaheurísticas}
\date{Curso 2017-2018}

%Quita la sangría
\setlength{\parindent}{0cm}


\begin{document}
	\maketitle

	\tableofcontents

	\newpage

	%p 56

%	\framebox[16cm][c]{\LaTeX}

	\section{Introducción del problema}
	\label{sec:introProblema}

	Para el problema de clasificación partimos de un conjunto de datos dado por una serie de tuplas que contienen los valores de atributos para cada instancia. Esto es una n-tupla de valores reales en nuestro caso.

	El objetivo del problema es obtener un vector de pesos que asocia un valor en el intervalo $[0,1]$ indicativo de la relevancia de ese atributo. Esta relevancia va referida a lo importante que es en nuestro algoritmo clasificador ese atributo a la hora de computar la distancia entre elementos.

	Resumiendo lo que tenemos es un algoritmo clasificador que utiliza el vector de pesos calculado para predecir la clase a la que pertenece una instancia dada. Este algoritmo clasificador es el KNN con k=1. Lo que hace es calcular según la distancia euclídea (o cualquier otra) la tupla más cercana a la que queremos clasificar ponderando cada atributo con el correspondiente peso del vector, es decir, la distancia entre dos elementos sería:
	$$d(e,f) = \sqrt{\sum_{i=0}^{n}w_i * (e_{i} - f_{i})}$$
	Donde e y f son instancias del conjunto de datos, w el vector de pesos y n la longitud de e y f que es la misma.

	La calificación que se le asigna al vector w depende de dos cosas: la tasa de aciertos y la simplicidad.

	La tasa de aciertos se mide contando el número de aciertos al emplear el clasificador descrito y la simplicidad se mide como el número de elementos del vector de pesos que son menores que 0.2, ya que estos pesos no son empleados por el clasificador, o lo que es lo mismo, son sustituidos por cero. Por lo tanto las calificaciones siguen las fórmulas:
	$$Tasa\_acierto = 100\cdot \frac{nº  \ aciertos}{nº \ datos} \ , \ Tasa\_simplicidad = 100\cdot \frac{nº \ valores \ de \ w \ < \ 0.2}{nº \ de \ atributos}$$
	$$Tasa\_agregada = \frac{1}{2}\cdot Tasa\_acierto + \frac{1}{2}\cdot Tasa\_simplicidad$$
	Cabe destacar que todas las tasas están expresadas en porcentajes, por lo tanto cuanto más cercano sea el valor a 100 mejor es la calificación.

	De esta forma a través del algoritmo que obtiene el vector de pesos para el conjunto de datos dado y el clasificador obtenemos un programa que clasifica de forma automática las nuevas instancias de datos que se introduzcan.


	\section{Introducción de la práctica}
	\label{sec:introPractica}
	
	En esta práctica he desarrollado algoritmos genéticos tanto estacionarios como generacionales con los dos operadores de cruce propuestos (aritmético y BLX) así como un algoritmo memético basado en el genético generacional.
	
	Al igual que en la práctica anterior el objetivo es ejecutar estos algoritmos sobre los conjuntos de datos dados para observar su comportamiento y realizar una comparativa entre los mismos. A los algoritmos mencionados anteriormente se les suman el 1NN con todos los pesos a uno, el greedy Relief y la búsqueda local (algoritmos implementados en la primera práctica).
	
	Igual que en la práctica anterior he realizado un procesamiento de los datos para eliminar tuplas repetidas, de forma que ya tenemos implementado el leave one out para conjuntos distintos en el KNN y además, he implementado una versión más rápida de KNN usando la librería NumPy con la intención de reducir tiempos en la búsqueda local, en los algoritmos genéticos y en los meméticos.
	
	En la práctica se desarrollará cómo he implementado los algoritmos genéticos (en sus dos variantes) incluyendo los operadores de cruce y mutación así como su adaptación a algoritmo memético.
	
	\newpage
	
	\section{Descripción común a todos los algoritmos}
	
	Los algoritmos empleados han sido el KNN, el algoritmo greedy Relief, la metaheurística de búsqueda local, un algoritmo genético estacionario, un algoritmo genético generacional y un memético basado en el algoritmo genético generacional.
	
	Estos algoritmos comparten ciertos métodos y operadores que pasaré a explicar en esta sección.
	
	Para empezar se debe destacar que la representación escogida para las soluciones es un vector de números reales, es decir, si n es el número de características:
	$$w\in \mathbb{R}^n \ t.q. \ \forall i \ con \ 0\leq i < n \ se \ tiene \ w_i \in [0,1]$$
	O lo que es lo mismo, un vector de tamaño n con todas las posiciones rellenas con números del intervalo [0,1].
	
	A estos números me referiré como pesos asociados a las características, ya que lo que nos indican es el grado de importancia de dicha característica a la hora de clasificar los datos, siendo 1 el máximo de relevancia y 0 el mínimo.
	
	Así mismo cabe destacar que nuestra intención en este problema es obtener una buena calificación de dicho vector de pesos. Esto lo medimos mediante las tasas de acierto y simplicidad que se definen como:
	$$Tasa\_acierto = 100\cdot \frac{nº  \ aciertos}{nº \ datos} \ , \ Tasa\_simplicidad = 100\cdot \frac{nº \ valores \ de \ w \ < \ 0.2}{nº \ de \ atributos}$$
	$$Tasa\_agregada = \frac{1}{2}\cdot Tasa\_acierto + \frac{1}{2}\cdot Tasa\_simplicidad$$
	
	La tasa de aciertos lo que nos mide es en un porcentaje cuántas instancias hemos clasificado correctamente mediante el algoritmo KNN usando el vector de pesos w.
	
	La tasa de simplicidad nos mide cuántos de los valores que tiene el vector de pesos son menores que 0.2. Esto se hace ya que, como imposición del problema, tenemos que si alguno de los pesos es menor que 0.2 no debemos usarlo, o lo que es lo mismo, debemos sustituirlo por un 0 en la función de la distancia que luego describiré. Midiendo esto obtenemos un dato de cuanto sobreajuste ha tenido nuestro algoritmo a la hora de obtener el vector de pesos. Cuantas menos características necesitemos para discernir la clase a la que pertenece una instancia de los datos, más simple será clasificar dicha instancia. Se expresa en porcentaje indicando 0 como ninguna simplicidad y 100 como la máxima simplicidad.
	
	De esta forma combinando ambas tasas obtenemos la tasa agregada que nos hace la media entre ambas tasas, de forma que le asignamos la misma importancia a acertar en la clasificación de las instancias y a la simplicidad en la solución. Cabe destacar que es imposible obtener una tasa de un 100\% a no ser que los datos se compongan únicamente de un punto ya que ello implicaría que la simplicidad ha de ser un 100\% (todos las posiciones del vector menores que 0.2) y por tanto la distancia sería 0 en todos los casos. De esta forma aspiraremos a una calificación lo mas alta posible pero teniendo en cuenta las restricciones de la función objetivo construida.
	
	Las funciones y operadores de uso común los he agrupado en un fichero llamado auxiliar.py. Este fichero contiene las funciones de lectura de datos, distancias, una función que devuelve el elemento más común de una lista, la norma euclídea, una función para dividir los datos en el número de particiones que queramos manteniendo el porcentaje de elementos de cada clase que había en el conjunto de datos original y el operador de mutación.
	
	\subsection{Generación de soluciones aleatorias}
	
	En los algoritmos genéticos y meméticos partimos de una población de soluciones aleatorias que generamos con una distribución uniforme, de forma que partimos en un inicio con una población de 30 individuos con valores en los vectores de pesos entre 0 y 1 generados de forma aleatoria.
	
	Nótese que en nuestro caso TAM\_POBLACION=30.
	
	\begin{algorithm}
		\caption{generaPoblacionInicial(longitud)}
		\begin{algorithmic}
			\STATE poblacion $\leftarrow$ [ ]
			\FOR{i=0 , ... , TAM\_POBLACION-1}
				\STATE cromosoma $\leftarrow$ [ ]
				\FOR{j=0 , ... , longitud-1}
					\STATE cromosoma $\leftarrow$ [cromosoma,uniforme(0,1)]
				\ENDFOR
				\STATE poblacion $\leftarrow$ [poblacion,cromosoma]
			\ENDFOR
			\RETURN poblacion
		\end{algorithmic}
	\end{algorithm}
	
	\subsection{Operador de cruce BLX-$\alpha$}
	
	Este operador de cruce se usa tanto en los algoritmos genéticos como meméticos.
	
	Se toman dos padres, de los que hallamos el elemento más grande de su vector de pesos y el más pequeño para poder obtener el máximo y mínimo de los dos. Esto nos va a dar un intervalo de valores para nuestro hijo. 
	
	El hijo se va a generar tomando valores aleatorios con una distribución uniforme que estén en el intervalo $[min\_padres - \delta , max\_padres + \delta]$, donde $\delta = max\_padres - min\_padres$. De esta forma podemos obtener el número que deseemos de hijos tan solo con dos padres, ya que los valores son aleatorios y por tanto los hijos obtenidos serán distintos.
	
	Nótese que en nuestro caso $\alpha=0.3$
	
	\begin{algorithm}
		\caption{cruceBLX(cromosoma1, cromosoma2)}
		\begin{algorithmic}
			\STATE hijo $\leftarrow$ [ ]
			\STATE max\_c1 $\leftarrow$ máximo(cromosoma1)
			\STATE max\_c2 $\leftarrow$ máximo(cromosoma2)
			\STATE min\_c1 $\leftarrow$ mínimo(cromosoma1)
			\STATE min\_c2 $\leftarrow$ mínimo(cromosoma2)
			\STATE max\_intervalo $\leftarrow$ máximo(max\_c1,max\_c2)
			\STATE min\_intervalo $\leftarrow$ mínimo(min\_c1,min\_c2)
			\STATE delta $\leftarrow$ $(max\_intervalo-min\_intervalo)\cdot \alpha$
			\FOR{i=0 , ... , longitud(cromosoma1)}
				\STATE hijo $\leftarrow$ $[hijo,uniforme(min\_intervalo - \delta, max\_intervalo + \delta)]$
			\ENDFOR
			\STATE Si hay alguna posición negativa en el hijo se trunca a 0.
			\STATE Si hay alguna posición mayor a 1 en el hijo se trunca a 1.
			\RETURN hijo
		\end{algorithmic}
	\end{algorithm}
	
	Nótese que hemos tenido que truncar las soluciones, ya que el valor $\delta$ utilizado en el algoritmo puede provocar que el hijo que obtengamos tenga valores fuera del intervalo [0,1], cosa que no tendría sentido para nuestro problema.
	
	\subsection{Operador de cruce Aritmético}
	
	El operador de cruce aritmético toma, igual que en el caso anterior, dos padres y devuelve un hijo. En este caso el hijo que obtenemos es único, ya que lo vamos a calcular haciendo la media posición a posición respecto a los dos padres. Esto nos va a garantizar que los hijos estén en el intervalo de definición, no como en el BLX.
	
	Un inconveniente que puede presentar este algoritmo es que vamos a tener mucha menos posibilidad de obtener valores muy cercanos a 0 o a 1, ya que al realizar la media siempre vamos a ir alejándonos de estos valores.
	
	\begin{algorithm}
		\caption{cruceAritmetico(cromosoma1,cromosoma2)}
		\begin{algorithmic}
			\STATE hijo $\leftarrow$ [ ]
			\FOR{i=0 , ... , longitud(cromosoma1)}
				\STATE hijo $\leftarrow$ [hijo,$\frac{cromosoma1[i]+cromosoma2[i]}{2}$]
			\ENDFOR
			\RETURN hijo
		\end{algorithmic}
	\end{algorithm}
	
	\subsection{Función de mutación}
	
	Esta función recibe como entrada un vector de pesos y una posición que es la que se desea mutar, devolviendo como resultado el vector de pesos ya mutado y la posición aumentada en una unidad (se usa para el algoritmo de búsqueda local aunque puede ignorarse).
	
	\begin{algorithm}[!h]
		\caption{mutacion(w,pos)}
		\begin{algorithmic}
			\STATE incremento = gauss(mu=0,sigma=0.3)
			\STATE posicion\_nueva = pos+1
			\STATE w[pos]+=incremento
			\STATE Truncar el vector w (0 si es negativo y 1 si es mayor que 1).
			\RETURN w,pos\_nueva
		\end{algorithmic}
	\end{algorithm}
	
	Esta función es usada en búsqueda local y en todos los genéticos y meméticos a la hora de realizar la mutación de los cromosomas.
	
	\newpage
	
	\section{Genético Estacionario}
	\label{sec:GE}
		
	\section{KNN}
	\label{sec:knn}
	
	\begin{algorithm}
		\caption{KNN(w,datos\_test,datos\_entrenamiento, etiquetas\_entrenamiento, etiquetas\_test, k, mismos\_conjuntos)}
		\begin{algorithmic}
			\STATE tam\_datos\_entrenamiento $\leftarrow$ longitud(datos\_entrenamiento)
			\STATE clases $\leftarrow$ []
			\FOR{i=0,...,longitud(datos\_test)}
			\STATE p $\leftarrow$ datos\_test[i]
			\STATE w\_m $\leftarrow$ Repetir el vector w tantas veces como datos haya en datos\_entrenamiento.
			\STATE p\_m $\leftarrow$ Repetir el vector p tantas veces como datos haya en datos\_entrenamiento.
			\STATE dist $\leftarrow$ $w\_m \cdot (p\_m - datos\_entrenamiento)^2$
			\IF{mismos\_conjuntos}
			\STATE dist[i] $\leftarrow$ $\infty$
			\ENDIF
			\STATE mins $\leftarrow$ Los k índices correspondientes a las distancias más pequeñas.
			\STATE clases $\leftarrow$ [clases, masComun(etiquetas\_entrenamiento[mins])]
			\ENDFOR
			\RETURN $\frac{Numero \ de \ elementos \ de \ clases \ que \ han \ acertado \ con \ respecto \ a \ etiquetas\_test}{longitud(etiquetas\_test)}$
		\end{algorithmic}
	\end{algorithm}
	
	Cabe notar que el número que devolvemos está entre 0 y 1, por lo que en los algoritmos de valoración debemos tener esto en cuenta para multiplicarlo por 100 y convertirlo en un porcentaje.
	
	\section{Relief}
	\label{sec:relief}
	
	\begin{algorithm}
		\caption{elementoMinimaDistancia(e,lista)}
		\begin{algorithmic}
			\STATE distancias $\leftarrow$ [ ]
			\FOR{l en lista}
			\IF{l!=e}
			\STATE distancias $\leftarrow$ [distancias, distancia(e,l,[1..1])]
			\ELSE
			\STATE distancias $\leftarrow$ [distancias, max(distancias)]
			\ENDIF
			\ENDFOR
			\STATE indice\_menor\_distancia $\leftarrow$ índice del elemento de menor valor del vector distancias.
			\RETURN lista[indice\_menor\_distancia]
		\end{algorithmic}
	\end{algorithm}
	
	\begin{algorithm}
		\caption{Relief(data)}
		\begin{algorithmic}
			\STATE w $\leftarrow$ vector de pesos a 0
			\FOR{elemento en data}
			\STATE clase $\leftarrow$ clase de elemento
			\STATE amigos $\leftarrow$ [ ]
			\STATE enemigos $\leftarrow$ [ ]
			\FOR{e en data}
			\IF{e!=elemento AND e[longitud(e)-1]==clase}
			\STATE amigos $\leftarrow$ [amigos, e]
			\ELSE
			\STATE enemigos $\leftarrow$ [enemigos, e]
			\ENDIF
			\ENDFOR
			\STATE amigo\_cercano $\leftarrow$ elementoMinimaDistancia(elemento, amigos)
			\STATE enemigo\_cercano $\leftarrow$ elementoMinimaDistancia(elemento, enemigos)
			\STATE resta\_enemigo $\leftarrow$ element-enemigo\_cercano
			\STATE resta\_amigo $\leftarrow$ element-amigo\_cercano
			\STATE w $\leftarrow$ w + resta\_enemigo - resta\_amigo
			\STATE $w_{max}$ $\leftarrow$ máximo de w
			\ENDFOR
			\FOR{i en [0..longitud(w)-1]}
			\IF{w[i]$<$0}
			\STATE w[i] $\leftarrow$ 0
			\ELSE
			\STATE w[i] $\leftarrow$ $\frac{w[i]}{w_{max}}$
			\ENDIF
			\ENDFOR
			\RETURN w
		\end{algorithmic}
	\end{algorithm}
	
	\newpage
	
	\section{Búsqueda Local}
	\label{sec:bl}
	
	\begin{algorithm}
		\caption{primerVector(n)}
		\begin{algorithmic}
			\STATE w $\leftarrow$ [ ]
			\FOR{i en [0..n-1]}
			\STATE w $\leftarrow$ [w, random.uniforme(0,1)]
			\ENDFOR
			\RETURN w
		\end{algorithmic}
	\end{algorithm}
	
	\begin{algorithm}
		\caption{busquedaLocal(data,k)}
		\begin{algorithmic}
			\STATE MAX\_EVALUACIONES $\leftarrow$ 15000
			\STATE MAX\_VECINOS $\leftarrow$ $20\cdot longitud(data[0])$
			\STATE vecinos $\leftarrow$ 0
			\STATE evaluaciones $\leftarrow$ 0
			\STATE posicion\_mutacion $\leftarrow$ 0
			\STATE w $\leftarrow$ primerVector(longitud(data[0]))
			\STATE valoracion\_actual $\leftarrow$ Valoracion(data,data,k,w)
			\WHILE{evaluaciones$<$MAX\_EVALUACIONES AND vecinos$<$MAX\_VECINOS}
				\STATE evaluaciones $\leftarrow$ evaluaciones+1
				\STATE vecinos $\leftarrow$ vecinos+1
				\STATE vecino, posicion\_mutacion $\leftarrow$ mutacion(w,posicion\_mutacion)
				\STATE valoracion\_vecino $\leftarrow$ Valoracion(data,data,k,vecino)
				\IF{valoracion\_vecino$>$valoracion\_actual}
					\STATE vecinos $\leftarrow$ 0
					\STATE w $\leftarrow$ vecino
					\STATE valoracion\_actual $\leftarrow$ valoracion\_vecino
					\STATE posicion\_mutacion $\leftarrow$ 0
				\ELSIF{posicion\_mutacion==longitud(w)}
					\STATE posicion\_mutacion $\leftarrow$ 0
				\ENDIF
			\ENDWHILE
			\RETURN w
		\end{algorithmic}
	\end{algorithm}	
	
	\section{Procedimiento de desarrollo de la práctica}
	\label{sec:procedimiento}

	
	\section{Resultados}
	\label{sec:resultados}
	
	\begin{table}[ht]
		\centering
		\resizebox{\textwidth}{!}{
			\begin{tabular}{| c | c | c | c | c | c | c | c | c | c | c | c | c |}
				\cline{2-13}
				\multicolumn{1}{c|}{} & \multicolumn{4}{|c|}{Ozone} & \multicolumn{4}{| c|}{Parkinsons} & \multicolumn{4}{|c|}{Spectf-Heart} \\ [0.5ex]
				\cline{2-13}
				\multicolumn{1}{c|}{} & \%\_clas & \%\_red & Agr. & T (seg) & \%\_clas & \%\_red & Agr. & T  (seg) & \%\_clas & \%\_red & Agr. & T  (seg) \\ [0.5ex] \hline
				Partición 1 & 71.8750 & 0.0000 & 35.9375 & 0.4403 & 76.3158 & 0.0000 & 38.1579 & 0.0546 & 70.5882 & 0.0000 & 35.2941 & 0.2561 \\ [0.5ex] \hline
				Partición 2 & 84.3750 & 0.0000 & 42.1875 & 0.4341 & 81.5789 & 0.0000 & 40.7895 & 0.0527 & 77.9412 & 0.0000 & 38.9706 & 0.2959 \\ [0.5ex] \hline
				Partición 3 & 71.8750 & 0.0000 & 35.9375 & 0.4329 & 94.7368 & 0.0000 & 47.3684 & 0.0543 & 67.6471 & 0.0000 & 33.8235 & 0.3161 \\ [0.5ex] \hline
				Partición 4 & 81.2500 & 0.0000 & 40.6250 & 0.4340 & 73.6842 & 0.0000 & 36.8421 & 0.0530 & 60.2941 & 0.0000 & 30.1471 & 0.2984 \\ [0.5ex] \hline
				Partición 5 & 85.9375 & 0.0000 & 42.9688 & 0.4484 & 76.7442 & 0.0000 & 38.3721 & 0.0586 & 66.2338 & 0.0000 & 33.1169 & 0.2715 \\ [0.5ex] \hline
				Media & 79.0625 & 0.0000 & 39.5313 & 0.4380 & 80.6120 & 0.0000 & 40.3060 & 0.0546 & 68.5409 & 0.0000 & 34.2704 & 0.2876 \\ [0.5ex] \hline
			\end{tabular}
		}
		\label{tabla1NN}
		\caption{Resultados 1NN}
	\end{table}
	
	\newpage
	
	\begin{table}[h!]
		\centering
		\resizebox{\textwidth}{!}{
			\begin{tabular}{| c | c | c | c | c | c | c | c | c | c | c | c | c |}
				\cline{2-13}
				\multicolumn{1}{c|}{} & \multicolumn{4}{|c|}{Ozone} & \multicolumn{4}{| c|}{Parkinsons} & \multicolumn{4}{|c|}{Spectf-Heart} \\ [0.5ex]
				\cline{2-13}
				\multicolumn{1}{c|}{} & \%\_clas & \%\_red & Agr. & T (seg) & \%\_clas & \%\_red & Agr. & T  (seg) & \%\_clas & \%\_red & Agr. & T  (seg) \\ [0.5ex] \hline
				Partición 1 & 71.8750 & 98.6301 & 85.2526 & 0.9703 & 78.9474 & 47.8261 & 63.3867 & 0.1254 & 75.0000 & 46.6667 & 60.8333 & 0.4568 \\ [0.5ex] \hline
				Partición 2 & 87.5000 & 97.2603 & 92.3801 & 1.0143 & 81.5789 & 26.0869 & 53.8330 & 0.1267 & 73.5294 & 53.3333 & 63.4314 & 0.6407 \\ [0.5ex] \hline
				Partición 3 & 71.8750 & 97.2603 & 84.5676 & 1.0385 & 86.8421 & 47.8261 & 67.3341 & 0.1263 & 67.6471 & 60.0000 & 63.8235 & 0.7070 \\ [0.5ex] \hline
				Partición 4 & 81.2500 & 95.8904 & 88.5702 & 1.0530 & 78.9474 & 52.1739 & 65.5606 & 0.1254 & 72.0588 & 31.1111 & 51.5850 & 0.6594 \\ [0.5ex] \hline
				Partición 5 & 85.9375 & 98.6301 & 92.2838 & 0.9963 & 72.0930 & 43.4783 & 57.7856 & 0.1180 & 72.7273 & 48.8889 & 60.8081 & 0.3986 \\ [0.5ex] \hline
				Media & 79.6875 & 97.5342 & 88.6109 & 1.0145 & 79.6818 & 43.4783 & 61.5800 & 0.1244 & 72.1925 & 48.0000 & 60.0963 & 0.5725 \\ [0.5ex] \hline
			\end{tabular}
		}
		\label{tablaReliefK1}
		\caption{Resultados Relief con K=1}
	\end{table}
	
	\newpage
	
	\begin{table}[h!]
		\centering
		\resizebox{\textwidth}{!}{
			\begin{tabular}{| c | c | c | c | c | c | c | c | c | c | c | c | c |}
				\cline{2-13}
				\multicolumn{1}{c|}{} & \multicolumn{4}{|c|}{Ozone} & \multicolumn{4}{| c|}{Parkinsons} & \multicolumn{4}{|c|}{Spectf-Heart} \\ [0.5ex]
				\cline{2-13}
				\multicolumn{1}{c|}{} & \%\_clas & \%\_red & Agr. & T (seg) & \%\_clas & \%\_red & Agr. & T  (seg) & \%\_clas & \%\_red & Agr. & T  (seg) \\ [0.5ex] \hline
				Partición 1 & 71.8750 & 34.2466 & 53.0608 & 186.6400 & 78.9474 & 26.0870 & 52.5172 & 9.0305 & 70.5882 & 35.5556 & 53.0719 & 52.5939 \\ [0.5ex] \hline
				Partición 2 & 76.5625 & 54.7945 & 65.6785 & 436.2914 & 81.5789 & 39.1304 & 60.3547 & 6.8897 & 82.3529 & 24.4444 & 53.3987 & 92.8766 \\ [0.5ex] \hline
				Partición 3 & 70.3125 & 30.1370 & 50.2247 & 185.3929 & 86.8421 & 17.3913 & 52.1167 & 4.3802 & 70.5882 & 57.7778 & 64.1830 & 120.7886 \\ [0.5ex] \hline
				Partición 4 & 82.8125 & 54.7945 & 68.8035 & 354.7868 & 78.9474 & 52.1739 & 65.5606 & 21.2622 & 63.2353 & 31.1111 & 47.1732 & 70.7596 \\ [0.5ex] \hline
				Partición 5 & 84.3750 & 39.7260 & 62.0505 & 224.3229 & 74.4186 & 30.4348 & 52.4267 & 9.3806 & 64.9351 & 44.4444 & 54.6898 & 90.2908 \\ [0.5ex] \hline
				Media & 77.1875 & 42.7397 & 59.9636 & 277.4868 & 80.1469 & 33.0435 & 56.5952 & 10.1886 & 70.3400 & 38.6667 & 54.5033 & 85.4619 \\ [0.5ex] \hline
			\end{tabular}
		}
		\label{tablaBLK1}
		\caption{Resultados Búsqueda Local con K=1}
	\end{table}
	
	
	\newpage
	
	\begin{table}[h!]
		\centering
		\resizebox{\textwidth}{!}{
			\begin{tabular}{| c | c | c | c | c | c | c | c | c | c | c | c | c |}
				\cline{2-13}
				\multicolumn{1}{c|}{} & \multicolumn{4}{|c|}{Ozone} & \multicolumn{4}{| c|}{Parkinsons} & \multicolumn{4}{|c|}{Spectf-Heart} \\ [0.5ex]
				\cline{2-13}
				\multicolumn{1}{c|}{} & \%\_clas & \%\_red & Agr. & T (seg) & \%\_clas & \%\_red & Agr. & T  (seg) & \%\_clas & \%\_red & Agr. & T  (seg) \\ [0.5ex] \hline
				1-NN & 79.0625 & 0.0000 & 39.5313 & 0.4380 & 80.6120 & 0.0000 & 40.3060 & 0.0546 & 68.5409 & 0.0000 & 34.2704 & 0.2876 \\ [0.5ex] \hline
				Relief & 79.6875 & 97.5342 & 88.6109 & 1.0145 & 79.6818 & 43.4783 & 61.5800 & 0.1244 & 72.1925 & 48.0000 & 60.0963 & 0.5725 \\ [0.5ex] \hline
				BL & 77.1875 & 42.7397 & 59.9636 & 277.4868 & 80.1469 & 33.0435 & 56.5952 & 10.1886 & 70.3400 & 38.6667 & 54.5033 & 85.4619 \\ [0.5ex] \hline
			\end{tabular}
		}
		\label{tablaGlobalK1}
		\caption{Resultados globales con K=1}
	\end{table}
	
	

\end{document}
